\documentclass[12pt]{article}
\usepackage{amsfonts}

\begin{document} %\pagestyle{empty}

\noindent \textbf{Report on ``Large-$N$ solution of the Heterotic
$CP(N-1)$ Model with Twisted Masses, by P.\ Bolokhov, M.\ Shifman and
A.\ Yung.}

String models of confinement have been studied over the last few years
in various four dimensional supersymmetric gauge theories, typically
by looking at small deformations of the Seiberg-Witten solutions for
the $\mathcal{N}=2$ theories. The effective worldsheet theories that one
obtains in this framework typically have $CP^{N}$ as a target
space and are very interesting field theoretic models on their own.
For example, the $(0,2)$ supersymmetric $CP^{N}$ non-linear
sigma model is conjectured to play a role in this context and the
authors of the paper to be reviewed have performed a detailed study of
its dynamics by using the standard large $N$ approximation.

The paper starts with a long review of results in the $(2,2)$
supersymmetric $\mathbb CP^{N}$ model: generalities about twisted
masses in Section 2, the formulation of the model in Section 3 (which
also includes the formulation of the $(0,2)$ case) and, mainly, the
study of the $(2,2)$ model at large $N$ in Section 4. All the results
in these Sections are well-known. It is certainly useful to review
them, because the study of the large $N$ heterotic model, which is the
original and interesting part of the paper (see below), uses similar
techniques and ideas. However, before publication, the authors will
have to cite the existing literature appropriately, since their
presentation may let the reader feel that a lot of results (in
particular in Section 4) are derived for the first time.

Important references for the $(2,2)$ model are [28] and [33] in the
reference list of the paper. All the relevant results on the large $N$
limit of the model with twisted masses are included in [33], including
the computation of effective potentials, saddle point equations
etc\ldots The analogues of Argyres-Douglas points are also discussed
in [33]. This covers a large part of Section 4. Other parts of Section
4, which deal with BPS states, are extensively covered in [28]. The
physics of the model is also discussed in full details in [28] and
[33] (and similar models, also including the relevant large $N$
techniques, in [34]).

Note that even though [28] and [33] are included in the list of
references, this is done in a very misleading way, giving the
impression that the content of these papers are only remotely related
to what the authors have done. However, essentially all the results
and formulas in Section 4 are included in [28] and/or [33], in exactly
the same or in slightly modified or generalized form.

Let us now discuss the core of the paper, Sections 5, 6 and 7. In
these Sections, the effect of breaking $(2,2)$ to $(0,2)$ is studied.
The strength of the breaking is governed by a parameter $u$ which
induces a new term in the effective potential (last term in equation
(5.1)). This small modification of the saddle point equations has
non-trivial and interesting consequences: the model breaks
supersymmetry for generic values of the parameters and has three
phases, a standard weakly coupled ``Higgs'' phase, a standard strongly
coupled phase with broken discrete symmetry, and an interesting
intermediate phase (which also lies at strong coupling) where the
discrete symmetry is restored and the ``photon'' is massless. This
phase diagram is deduced from a detailed analysis of the saddle point
equations. The authors have studied special limits (small $u$ and
large $u$) and have supplemented their analytic calculations with
numerical checks.

Overall I find the picture that emerges quite convincing. To my
knowledge, it is the first such study of the properties of the
heterotic $(0,2)$ model, and even though the results are not very
surprising they are interesting and non-trivial.

The paper ends by addressing a few additional questions in Section 8.
The results of this last Section are again well-known. I feel that
this Section, which is far less interesting than the previous ones,
would be better reduced and put in an Appendix.

My conclusion is that the paper contains three interesting Sections
discussing the physics of the $(0,2)$ $\mathbb CP^{N}$ model. This is
an original and good-quality research that deserves to be published.
On the other hand, the other (long) Sections of the paper only review
well-known facts. Before publication, the existing literature must be
more appropriately cited. E.g., in the Introduction, when the author
announce ``As a warm up exercice we analyze this model in the limit of
vanishing heterotic deformation\ldots,'' they should mention the
references [28] and [33] and clearly state that their original results
concern the heterotic version of the model.

I will of course be happy to check the modified version of the paper
when it is resubmitted.

\end{document}
