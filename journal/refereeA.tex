\documentclass[a4paper,12pt]{article}

\begin{document}

\section*{Report on DN10622}

Title: Large-$N$ solution of the heterotic $CP(N-1)$ model with
twisted masses\\

Authors: Pavel A. Bolokhov, Mikhail Shifman, Alexei Yung\\

Physical Review D\\

This is a very good paper with very interesting results and should be
published in Physical Review D, after some clarifications. Most of the
questions and suggestions I make below are minor things. There is one
crucial point that has to be clarified in my opinion (remark 3. to
section 4). I apologize in advance for any misunderstanding on my side
and I hope the authors see my questions/suggestions as an input for
``fine-tuning'' this interesting paper from an reader which is not
occupied by and involved in the calculations and the elaborations of
the details.

\subsection*{Section 2}

\begin{enumerate}

\item \underline{equ. (2.2)}: Can one shortly comment here why the
$Z_N$ symmetric masses are of interest for the microscopic 4D
gauge theory?

\item Below equ. (2.2) it is stated that ``Without loss of
generality...'' one can choose $m\in \mathcal{R}_{+}$. On the other
hand it is explicitly shown in section 4 (equ. 4.24 and following)
that a complex $m$ allows for additional phenomena.

\end{enumerate}

\subsection*{Section 3}

\begin{enumerate}

\item \underline{equ. (3.13)}: All geometrical quantities are defined
except the Christoffel symbol. However, the whole subsection is
otherwise very elaborate, I will come back to this.

\item \underline{equ. (3.23)}: Below this equation it is written ``...
does not break ${\cal N}=2$ supersymmetry". I guess the notation
(consistent with the actual paper) should be ``${\cal N}=(2,2)$"
(though ${\cal N}=2$ was used in previous works). One should perhaps
say that this is so because the background (3.22) is invariant under
$4D\ \ {\cal N}=1$ susy (this explains very simple the stated fact,
which is not obvious from (3.24)).

\end{enumerate}

In general I would like to mention that the review of the geometric
formulation (also in the deformed case) takes a lot of space and
somehow breaks the flow of the paper. Especially since these results
are not needed in the bulk of the paper were the new results are
discussed (equ. (8.1) is sufficient to understand the subsequent
discussion).

%\subsection*{Section 3, continued}

\begin{enumerate}

\item \underline{equ. (3.35)}: Below (3.35) it is stated that for $m_l
=0$ this reduces to the result of [13]. But is (3.35) for general
$m_l$ not equivalent to the result of [16]?

\item \underline{equ. (3.36)}: The splitting in the boson and fermion
mass shows that susy is spontaneously broken at the classical level.
This is not mentioned at this point, though later (below (3.42) and
end of page 28) this very fact is mentioned with reference to [16].
One should refer at these places (also) to the explicit demonstration
of this fact in the actual paper, i.e. equ. (3.36).

\end{enumerate}

\subsection*{Section 4}

\begin{enumerate}

\item \underline{equ. (4.2)}: Below (4.2) it is just said ``we dropped
the gauge field $A_k$''. It gives the impression here that this is
justified by the large $N$ approximation. The integration over $A_k$
gives contribution to the kinetic terms, especially it generates a
kinetic term for the a priori auxiliary field $A_k$ [6]. This is not
relevant for the following discussion in the paper and therefore $A_k$
can be ``dropped'' here (but it appears in section 7). I think it
would avoid misunderstandings if one makes this distinction clear at
this point.

\item \underline{equ. (4.18)}: It is stated that (4.18) follows from
(4.17) in the large $N$ limit. This is not obvious to me, as far as I
can see this is only true if $\sqrt{2}\sigma \sim \Lambda$ and $m <
\Lambda$ , which is justified a posteriori.

\item \underline{equ. (4.19)}: This equation is very puzzling, in my
opinion something is wrong here. It is stated that (4.19) is exact and
coincides with [35,36,7,37,28]. But the exact pendant of (4.19) is a
stronger equation, namely for the complex expression without taking
the absolute value, as given in equ. (4.26) and noted in footnote 7.
In (4.26) it is referred to it as the ``complexified version'' but it
is not clear to me how this complexified version should follow from
the results given in the paper (as far as I can see everything is
valid to this point also for complex $m$, this cannot be the
explanation). In particular, the solutions (4.22) are only a subset of
the solutions of the too weak equation (4.20). For example with
$\epsilon=m/\Lambda<1$ there is a continuum of solutions
$\sqrt{2}\sigma =
(\epsilon^N+\sqrt{1-\epsilon^{2N}\sin^2(N\phi)})^{1/N}e^{i\phi}$ to
(4.20), also large $N$ does not help here. (4.22) is of course the
correct set of vacua, so (4.20) cannot be the whole story, something
is missing here. Perhaps one has to be more careful with the branch
cut contributions in the effective potential, somehow the holomorphic
structure is not evident. I don't see an obvious solution to this from
the equations given in the paper, but this point has to be clarified.
As far as I can say this problem does not "propagate" to the
subsequent investigations, and the resolution at this point would be
sufficient. This is also confirmed by the numerical calculations,
though I of course cannot say anything about the detailed
implementation thereof.

\item \underline{equ. (4.25)}: Here one has the same problem as just
described under 3. Also at the AD point in addition to the correct
solution $\sigma =0$ exists a continuum of additional solutions to
(4.20).

\end{enumerate}

\subsection*{Section 5}

\begin{enumerate}

\item \underline{equ. (5.1)}: Above (5.1) one could refer to (3.31) as
the only difference to the considerations of section 4 thus leading to
(5.1)

\item \underline{equ. (5.2)}: If I understand correctly $\omega$ is
the one given in (3.30) (if this is the case one could give a
reference to (3.30) here). However this is somehow puzzling, see 4., I
guess there is a renormalization involved. Perhaps one can define
$\omega$ ($u$) more clearly.

\item \underline{equ. (5.9)}: I think $\beta_{ren}$ has to be replaced
by $\beta^{(0)}_{ren}$.

\item \underline{equ. (5.13)}: Below (5.13) it is stated that $n$
vanishes in all orders of $u$. Therefor the same is true for
$\beta_{ren}$. On the other hand $u\sim \beta \delta$ which would mean
that also $u$ vanishes (see 2.).

\item \underline{equ. 5.14}: I think there is an ``$=$'' missing in
the second condition in (5.14).

\end{enumerate}

\subsection*{Section 6}

\begin{enumerate}

\item \underline{equ. (6.2)}: Same question as in 4. of section 5.

\item \underline{page 28}: Last sentence, see 4. of section 3.

\item \underline{page 29}: In the third paragraph it is discussed that
susy is restored by quantum effects for $m=\Lambda$, with reference to
[41]. Is the ``remarkable'' observation here that this happens only
for this particular point? Also for the undeformed ${\cal N}=(2,2)$
models susy is restored (negative FI-coupling) at the quantum level
[7,28].

\item \underline{Fig.3}: If I did not overlook it, the dashed line in
the graph is not explained.

\end{enumerate}

\subsection*{Section 8}

\begin{enumerate}

\item \underline{equ. (8.1)}: The geometric formulation as given here
would be sufficient for the purpose of the paper. See below 2. of
section 3.

\item \underline{equ. 8.7}: if I am not mistaken ${\cal{H}}$ is the
same as $S$ of reference [45], it would be easier for the reader if
this would be mentioned here.

\item \underline{equ. 8.8}: In (8.8) ${\cal W}{'}$ should be ${\cal
W}{'}_{mirror}$ I guess. The index $Y$ on $\psi$ is a bit irritating.
I understand that it should be the fermion in the superfield $Y$, but
in (8.5) and (8.7) this notation was not used.

\item \underline{equ. (8.15)}: In the paragraph above (8.15), as it is
formulated, one gets the impression that the superpotential for
$CP^{N-1}$ models is ``engineered'' to reproduce (anomalous) Ward
identities. For this models the superpotential has been
\emph{computed} exactly [36].

\end{enumerate}

To summarize I want to say that I enjoyed very much this very
interesting paper. The only general issue that I see (besides 3. of
section 4) is the length of the paper. Given the length of the paper I
find that the review of the geometric formulation in section 3 breaks
really the flow of the paper. This formulation is not needed in the
bulk of the paper and was given elsewhere by the same authors in
better detail. Perhaps it would tighten the paper a bit if this part
would be placed in the appendix. As always, things also depend on the
personal taste and I hope the authors don't misunderstand my
suggestions/questions as disrespect for their right to put things
according to their taste. I hope the authors clarify the necessary
points so that this very good paper can be published.

\end{document}
