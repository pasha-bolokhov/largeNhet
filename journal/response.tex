\documentclass[a4paper,12pt]{article}

\begin{document}

{
\flushleft
Dr. Ansar Fayyazuddin\\
Assistant Editor\\
Physical Review D
}

\vspace{1.5cm}
%\section*{Report on DN10622}
\centerline{
\Large\bf
Response to Reports on DN10622
}

\vspace{0.7cm}
Dear Dr. Fayyazuddin,
\vspace{0.45cm}

Thank you for your message of March 8 containing referees' reports with
regards to the above paper.
We are also grateful to the referees for careful reading of the manuscript
and a large number of suggestions aimed at improving our presentation. 
We agree with practically all suggestions and tried to do our best in revising 
the paper, without compromising its message, which, as both referees agree is 
scientifically sound.

In summary, we made all changes recommended by referee 2, adding explanations
where necessary, expanding derivations and correcting typos in some formulas.
To streamline our presentation, we moved a part of the main text devoted
to the geometric formulation in Appendix E. Now the geometric formulation is only 
marginally used in Sect. 8, while all necessary definitions and facts are 
referred to this Appendix.

In particular, we would like to comment on two main remarks of the second
referee. The first one is the remark 3 to Sect. 4. We included an explanation
on p.16 that the phase factor in (4.22) does not follow from equation
(4.19). It appears due to anomalous breaking of $U(1)_R$ symmetry down to
$Z_{2N}$. We also include the complexified version of (4.19) in (4.23).

Second is the remark 2 and 4 to Sect. 5 about relation between
deformation parameters $\omega = \delta \sqrt{2\beta}$ in the strong
coupling phase, where renormalized $\beta$ is equal to zero. As the second
referee points out renormalization is involved here. The above relation
is only classical definition valid for bare parameters and $\beta$ is
the bare coupling constant, which is non-zero. Once the renormalization is
taken into account the convenient deformation parameter is $\omega$,
which now (in the revised version) we introduce in (3.9) as independent
parameter. The point is that two coupling constants of the model,
namely $\beta$ and $\omega$ diagonalize the renormalization group
equations and therefore it is convenient to use these parameters,
rather than others.

We would like to mention that the work done  by this referee in reading 
our manuscript is remarkable. We would like to thank the referees in the 
Acknowledgments.

We also added 4 references (in Section 8), with the appropriate comments. 
For convenience of the readers, we added Appendix C in which we compile and 
compare various definitions of the heterotic deformation parameters one can 
find in the literature. The large-$N$ scaling laws for all these definitions 
are summarized here.

As regards to the first referee, we changed our Introduction and Section 4
in such a way that now it is perfectly clear (hopefully) what was known in 
the literature previously and is reviewed in our paper to prepare the theoretical 
set up for original studies, on the one hand, and what is our original 
contribution, on the other. We enhanced the citation of Refs. 28, 33 and 34.

With all due respect we  do not accept, however, the statement of the first 
referee that Sects. 1 to 4 and Sect. 8 are not needed and should be 
dramatically reduced. Sections 1 and 2 present introduction. They occupy 4 pages, 
which seems justified given a large number of issues we address in the original 
part of this work. Section 3, presenting our basic model and describing relevant 
physics, is only 4 pages long. It is barely enough to introduce all relevant 
notation.

Section 4, besides reviewing our previous results, introduces all necessary
machinery and theoretical methods needed for the large-N solution of the 
heterotic model with twisted masses. It is 9 pages long, which is indeed, 
somewhat on a long side. However, without explicit discussion of the machinery 
and methods used in our original studies, the paper looses its self-contained 
nature, and could not be read (nor understood, as a matter of fact), without 
preliminary studies of at least three previous papers.

With all due respect we disagree with the first referee with regards to
Section 8. Sect. 8.1 presents a new result: construction of the mirror 
representation for weakly heterotically deformed CP model with twisted masses. 
Section 8.2 explicitly demonstrates the difference between the 
Veneziano-Yankielowicz-type effective Lagrangian (8.14) and the exact large-N 
solution at order $u^2$. Sect. 8.3. sheds new light on the determination of 
the curve of marginal stability (its form was known previously, but we use a new 
method to confirm previous results).

The revision we undertook is extensive. At the same time both referees note that 
new results we obtained on physics of the heterotically deformed CP models with 
twisted masses are solid and interesting. We sincerely hope that the revised 
version we would like to submit now will be positively evaluated and the paper 
can be published without further delay.

\vspace{0.6cm}
Best regards

{\flushleft
Pavel A. Bolokhov, Mikhail Shifman, Alexei Yung
}

\end{document}
