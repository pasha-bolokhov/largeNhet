\documentclass[epsfig,12pt]{article}
\usepackage{epsfig}
\usepackage{graphicx}

\def\beq{\begin{equation}}
\def\eeq{\end{equation}}
\def\beqn{\begin{eqnarray}}
\def\eeqn{\end{eqnarray}}
\def\Tr{{\rm Tr}}
\newcommand{\nfour}{${\cal N}=4\;$}
\newcommand{\ntwo}{${\mathcal N}=2\,$}
\newcommand{\none}{${\mathcal N}=1\,$}
\newcommand{\ntt}{${\mathcal N}=(2,2)\,$}
\newcommand{\nzt}{${\mathcal N}=(0,2)\,$}
\newcommand{\tth}{${T^3}\,$}
\newcommand{\ca}{{\mathcal A}}
\newcommand{\cell}{{\mathcal L}}
\newcommand{\cw}{{\mathcal W}}
\newcommand{\cs}{{\mathcal S}}
%\newcommand{\vp}{\varphi}
%\newcommand{\pt}{\partial}
%\newcommand{\ve}{\varepsilon}
%\newcommand{\gs}{g^{2}}
%\newcommand{\qt}{\tilde q}
%\renewcommand{\theequation}{\thesection.\arabic{equation}}
%%%%%%%%%
\usepackage{latexsym}
\usepackage{amsmath}
\usepackage{amssymb}
\usepackage{relsize}
\usepackage{geometry}
\geometry{letterpaper}
\usepackage{color}
\usepackage{showlabels}
\begin{document}

%%%%%%%%%%%%%%%%%%%%%%%%%%%%%%%%


\section{Notes with regards to Sect. 8.3 SBY}

{\bf Subsect. 8.3}

\subsection{When the \boldmath{$n$} fields can be considered as solitons}
\label{wtnfcb}

Long ago Witten showed \cite{W79} that the $n$ fields in fact describe
kinks interpolating between two neighboring vacua picked up from the set of $N$ degenerate 
supersymmetric vacua
of the \ntt sigma model. The above statement refers to the model with no twisted masses.
(See Fig.~\ref{nkin}).
Here we will discuss the physical status of these states, and the BPS spectrum at large,
as the twisted mass parameter evolves towards large values, $|m|/\Lambda \gg1$.
$N$ will be assumed to be large so that we can use the large-$N$ solutions.

In the undeformed \ntt theory two distinct regimes are known to exist.
\begin{figure}
\epsfxsize=5cm
%%\centerline{\epsfbox{nsigma22.eps}}
%\centerline{\epsfbox{nkinks.eps}}
\centerline{\resizebox{5cm}{!}{\input{nkinks_p.pstex_t}}}
\caption{\small The kinks are represented by the $n$ fields at $|m_i|<\Lambda$. }
\label{nkin}
\end{figure}
At large $m$
the theory is in the Higgs regime, while at small $m$ it is in the
strong coupling regime.
There is no phase transition between the two regimes, since 
the global $Z_N$ symmetry of the model is spontanepously broken in both.
An apparent dsicontinuity of, say, the derivative of the vacuum expectation of $\sigma$ at
$m=\Lambda$ (see Fig.~\ref{fig22nsigma}) is an artifact of the $N\to\infty$ limit.
However, the BPS spectrum experiences a drastic change in passing from small to large $m$.
In particular, 
at $|m|<\Lambda$, the masses $M_i$ of the $n^i$ fields in the vacuum $|0\rangle$
are
\beq
M_i^2 = 2\left| \Lambda - \frac{m_i}{\sqrt 2}\right|^2\,,\qquad i=0,1, ..., N-1\,.
\label{nkma}
\eeq
This follows from Eq.~(\ref{bee31}). The mass degeneracy of the kink $N$-plet is gone
since the twisted mass terms (\ref{two}) break the SU$(N)$ symmetry of the model leaving intact only
U(1)$^{N-1}$ and \zn-symmetries.

On the other hand, at large $m$, in the Higgs regime at weak coupling,  the $n$ fields no longer describe solitons;
rather they represent elementary excitations. In each of the $N$ vacua
there are $2(N-1)$
real elementary excitations. Say, consider the vacuum in which $n^0$ develops a VEV.
The phase of $n^0$ is eaten by the Higgs mechanism providing the mass to the photon field.
The modulus of $n^0$ is excluded from dynamics by the constraint $\left|n^0\right|^2=2\beta$.
The elementary excitations are described by $n^1,\,n^2,\, ... ,\, n^{N-1}$.
The same equation~(\ref{bee31}) implies that the masses of these elementary excitations are\,\footnote{In fact, 
equation $M_k= \left| m_0 - m_k\right|$ is exact.}
(at large $N$)
\beq
M_k^2 = \left| m_0 - m_k\right|^2=
\left(m\, \frac{2\pi k}{N}
\right)^2
\,,\qquad k=1, 2, ..., N-1\,.
\label{nkmap}
\eeq
since the vacuum expectation of $\sigma$ in the vacuum under consideration is $\sigma\approx m_0\equiv m$.
For the values of $k$ that do not scale with $N$, each such mass in (\ref{nkmap}) scales as $1/N$ at large $N$.
This is the consequence of our $Z_N$-symmetric choice of the twisted mass parameters
(\ref{two}). 

%If $k\sim N$ then $M\sim m$.

What about the kink masses in the Higgs regime? (For brevity we will refer to them as
the weak-coupling regime (WCR) kinks, although this is not quite precise.
Indeed, if $m$ is in the Higgs domain but $\sim {\rm a\,\, few}\times\Lambda$, the coupling constant is 
of order 1.) The masses of the WCR
kinks
can be found from the Veneziano--Yankielowicz
effective superpotential \cite{HaHo},
which is exact in the model under consideration,
\beq
\cw_{\rm VY} 
=
 \frac{1}{4\pi}
\,\sum_{i=0}^{N-1}\left(\sqrt{2}\,\sigma -m_i
\right)\ln\frac{\sqrt{2}\,\sigma -m_i}{\Lambda}-\frac{N}{4\pi}\sqrt{2}\,\sigma\,.
\label{widu}
\eeq
For definiteness $N$ is assumed to be even.
%;
%then Eq.~(\ref{widu}) can be rewritten in the equivalent form
% \beq
%\cw_{\rm VY} 
%=
% \frac{1}{4\pi}
%\,\sum_{i=0}^{N-1}\left(\sqrt{2}\,\sigma + m_i
%\right)\ln\frac{\sqrt{2}\,\sigma + m_i}{\Lambda}-\frac{N}{4\pi}\sqrt{2}\,\sigma\,.
%\label{widum}
%\eeq
One should be careful with the logarithmic function. It is obvious that (\ref{widu})
is defined up to $im_p/2$ times any integer number.
For the kinks interpolating between the  vacua $|0\rangle$ and  $|p\rangle$  one has \cite{HaHo}
\beq
M_{\rm kink} 
=
 2\left|\Delta \cw\right|= 2\left|\cw_{\rm VY} (\sigma_0) - \cw_{\rm VY} (\sigma_p)+\frac{i}{2}\left(m_0-m_p\right) q  \right|\,,
   \label{widup}
\eeq
where the logarithmic function in (\ref{widu}) is defined with the cut along the negative semi-axis, and $q$ is an arbitrary integer.
Since the vacuum values of sigma satisfy
$$(\sqrt{2}\sigma)^N = m^N + \Lambda^N$$ and $|m/\Lambda| >1$, we 
have $\sqrt{2}\sigma_p = m_p$, with the exponential accuracy.
If  $\sigma_0=m_0=m$ and $|p|<N/4$, in calculating  $\cw_{\rm VY} (\sigma_0, \,\sigma_p)$
we do not touch the cut of the logarithm, and then we can use the following expression\,\footnote{One might be tempted to use Eq.~(\ref{usefulformula}) here, but this does not work
since in this expression we have a different argument of the logarithm.}
\beqn
\cw_{\rm VY}(\sigma)
&=&
  \frac{1}{4\pi}\,\sqrt{2}\sigma\,   \ln\left(\frac{\sqrt{2}\sigma}{e\Lambda}\right)^N
+\frac{1}{4\pi\,N }\,\sqrt{2}\sigma\,F_N(y)\,,
\nonumber\\[3mm]
F_N(y) 
&\equiv&
 \sum_{k=1}^\infty \,\,\frac{N}{k\, (kN-1)}\, y^k
\nonumber\\[3mm]
 y 
&=& 
\left(\frac{m}{\sqrt{2}\,\sigma}\right)^N\,.
\label{chet1}
\eeqn
Note that $F_N$ is finite at $N\to \infty$ and $|y|\leq 1$. 
As a result, at  $|p|<N/4$ we arrive at
\beq
M_{\rm kink} 
=
\frac{N\,m}{\pi}\,\sin\frac{\pi\, |p|}{N}\left|\ln\frac{m}{e\Lambda} +2\pi i\frac{q}{N}\,.
\right|\,.
\label{chet4}
\eeq
The lightest mass is obtained by setting $q=0$,
\beq
M_{{\rm kink}*} 
=
\frac{N\,m}{\pi}\,\sin\frac{\pi\, |p|}{N}\left|\ln\frac{m}{e\Lambda} 
\right|
\label{chet4w}
\eeq
At $p\sim N^0$ and
$|m/\Lambda |\gg1$ the kink mass does not scale with $N$. In addition,  it is enhanced by 
the large logarithm $\ln m/e\Lambda\sim (Ng^2)^{-1}$ compared to the masses of the elementary excitations.
Each kink has a tower of excitations on top of it, corresponding to $q\neq 0$.
The existence of excitations is due to the fact that
in additition to  the topological charge, the kinks can  have U(1) charges \cite{Dor}. 
Excitations decay on the curve of marginal stability (CMS) $|m| =e\Lambda$, see \cite{Olmez}.


Now consider $|p|$ larger than $N/4$. Equation (\ref{chet1}) is no longer valid. Instead, in the
limit $N=\infty$ we have ($\frac{1}{4}\leq x\leq \frac{3}{4}$)
\beq
\cw_{\rm VY}(\sigma_p=m_p)\to \frac{m_p}{4\pi}\,\ln\left(\frac{m}{e^{f_1(x)+i f_2(x)}\,\Lambda }\right)^N
\label{chet2}
\eeq
where
\beqn
f_1 
&=&
 -\cos 4\pi\,x\,,\qquad f_2 = \left\{\begin{array}{l}
4\pi\left(x-\frac{1}{4}\right) +\sin 4\pi\,x\,,\qquad \frac{1}{4}\leq x\leq \frac{1}{2}\,,\\[3mm]
4\pi\left(x-\frac{3}{4}\right) +\sin 4\pi\,x\,,\qquad \frac{1}{2}\leq x\leq \frac{3}{4}
\end{array}
\right.
\label{chet2w}
\\[3mm]
x
&=& \frac{p}{N}\,.
\label{chet2ww}
\eeqn 
For kinks interpolating between the  vacua $|0\rangle$ and  $|p\rangle$  with $\frac{N}{4}\leq p\leq \frac{3N}{4}$
we thus get
\beqn
M_{\rm kink} 
&=&
\frac{N\,m}{\pi}\left(\sin \pi\,x\right)
\nonumber\\[3mm]
&\times&
\left|
\ln\frac{m}{\Lambda e^{(1+f_1)/2}} \, - f_2\,\cot \pi x -\frac{i}{2}\left[f_2 +(1-f_1)\cot\pi x
\right]+2\pi i \frac{q}{N}
\right|
\nonumber\\
\label{chet4ww}
\eeqn
In particular, the
 ``most composite"  kink corresponding to $p=N/2$ and $x=1/2$.
For such kinks 
\beq
M_{\rm kink} 
=
\frac{N\,m}{\pi}\,\left|\ln\frac{m}{\Lambda} +2\pi i\frac{\tilde q}{N}
\right|\,.
\label{chet3}
\eeq
The tower of excitations collapses at $|m| = \Lambda$. This is another CMS (to which the AD point belongs).
Generally speaking, there are $N/2$ curves of marginal stability
corresponding to distinct values of $p$ in the interval $(N/4, 3N/4)$
and such values of $m$ for which the real part in the second line in (\ref{chet4ww}) vanishes.
They are presented by co-axial circles in the $m^N$ plane with radii $r^N$
where $r\in [1, e]$. There is a cut in the $m^N$ plane along the negative semi-axis \cite{Olmez}.
At $|m|<\Lambda$ all excitations decay, and the kinks are represented by the $n$ fields.

So far we ignored the heterotic deformation.
Since the latter generally speaking breaks 
supersymmetry (spontaneously), the analysis presented above is no longer valid.
However, for small heterotic deformation, the overall picture is likely to
stay intact as long as we are in the $Z_N$ broken phase. 
We still have $N$ degenerate vacua (albeit they are at nonvanishing energy),
and solitons of various kinds which interpolate between them.
The BPS saturation-based consideration with regards to
the soliton masses and  CMS is no longer applicable,
and these masses and  curves will be distorted compared to the \ntwot case.
These distortions should be small, however.


\vspace{1cm}

{\bf Probably, it makes sense to }

1) check equations;

2) $\tilde q$ seems to contain 1/4

3) For each $p$ take such value of $q$ which minimizes the kink mass.
Then plot $M_{\rm kink*}$ as a function of $x$
for some value of $m/\Lambda$ and $N$. Say,
$m/\Lambda = 4$ or 5.



\small
\begin{thebibliography}{99}
\itemsep -2pt

 \bibitem{Olmez}
S.~Olmez and M.~Shifman,
  %``Curves of Marginal Stability in Two-Dimensional CP(N-1) Models with
  %Z_N-Symmetric Twisted Masses,''
  J.\ Phys.\ A  {\bf 40}, 11151 (2007)
  [arXiv:hep-th/0703149].
  %%CITATION = JPAGB,A40,11151;%%

  
  
\end{thebibliography}

\end{document}
