\documentclass[epsfig,12pt]{article}
\usepackage{epsfig}


\def\beq{\begin{equation}}
\def\eeq{\end{equation}}
\def\beqn{\begin{eqnarray}}
\def\eeqn{\end{eqnarray}}
\def\Tr{{\rm Tr}}
\newcommand{\nfour}{${\cal N}=4\;$}
\newcommand{\ntwo}{${\mathcal N}=2\,$}
\newcommand{\none}{${\mathcal N}=1\,$}
\newcommand{\ntt}{${\mathcal N}=(2,2)\,$}
\newcommand{\nzt}{${\mathcal N}=(0,2)\,$}
\newcommand{\tth}{${T^3}\,$}
\newcommand{\ca}{{\mathcal A}}
\newcommand{\cell}{{\mathcal L}}
\newcommand{\cw}{{\mathcal W}}
\newcommand{\cs}{{\mathcal S}}
%\newcommand{\vp}{\varphi}
%\newcommand{\pt}{\partial}
%\newcommand{\ve}{\varepsilon}
%\newcommand{\gs}{g^{2}}
%\newcommand{\qt}{\tilde q}
%\renewcommand{\theequation}{\thesection.\arabic{equation}}

\begin{document}

%%%%%%%%%%%%%%%%%%%%%%%%%%%%%%%%


\section{Notes with regards to SBY}

\subsection{Different effective Lagrangians}

In this section we will comment on the relation between the
effective Lagrangian derived in Sect.~\ref{hecpnsm} from the large-$N$ expansion
 and the Veneziano--Yankielowicz
effective Lagrangian based on  anomalies and supersymmetry.
For simplicity we will set $m_i =0$ in this section. 
Generalization to  $m_i \neq 0$ is straightforward. We  assume the heterotic deformation to be small, $u\ll1$.

The $1/N$ expansion allows one to derive an honest-to-god effective
Lagrangian for the field $\sigma$, valid both in its kinetic and potential parts. The  leading order in
$1/N$ in the potential part is determined by the diagram depcited in Fig.~\ref{dopone}
which gives
\begin{figure}
\epsfxsize=6cm
\centerline{\epsfbox{esigma}}
\caption{\small
Kinetic term for the  for  $\sigma$ field. }
\label{dopone}
\end{figure}
%%%%%%%%%%%%%
\beq
{\mathcal L}_{\rm kin} =\frac{N}{4\pi}\,\frac{1}{2|\sigma|^2}\,| \,  \partial_\mu\sigma \,|^2
\,.
\label{fdop1}
\eeq
The  virtual $\xi$ momenta saturating the loop integral are of the order of the $\xi$
mass $\sqrt{2}|\sigma|$. Up to a numerical coefficient this result is obvious since the field
$\sigma$ has mass-dimension 1.

The potential part following from calculations in Sect.~\ref{hecpnsm}
is 
\beq
{\mathcal L}_{\rm pot} = \frac{N}{4\pi}\left\{\Lambda^2 +2|\sigma|^2\left[
\ln\frac{2|\sigma|^2}{\Lambda^2}-1+u\right]
\right\}.
\label{fdop2}
\eeq
All corrections to (\ref{fdop1}) and  (\ref{fdop2}) are suppressed by powers
of $1/N$. For what follows it is convenient to introduce a dimensionless variable
\beq
\cs = \frac{ \sqrt{2}\sigma}{\Lambda}\,.
\eeq
Then the large-$N$ effective Lagrangian of the $\sigma$ field takes the form
\marginpar{\tiny NEPRAVILNO?}
\beq
{\mathcal L}_{\rm eff} =\frac{N}{4\pi}
\left\{
\frac{1}{\,2\, |\cs|^2}\,\,
 | \partial_\mu\cs \, |^2
+\Lambda^2
\left[1 +|\cs |^2\left( \ln |\cs|^2 -1+u\right)\right]
\right\}.
\label{fdop3}
\eeq

On the other hand, 
the Veneziano--Yankielowicz method \cite{VYan} produces 
an effective Lagrangian in the Pickwick sense. It realizes, in a superpotential,  the  anomalous Ward identities of the 
underlying theory
and other symmetries, such as supersymmetry, and gives no information on the kinetic
part. In the CP$(N-1)$ models
the Veneziano--Yankielowicz superpotential ${\mathcal W}_{\rm VY} = \Sigma \ln \Sigma $ (for twisted superfields)
was obtained in \cite{AdDVecSal,ChVa,W93}. 
In terms of the scalar potential for the $\sigma$ field
the Veneziano--Yankielowicz  construction has the form
\beq
V_{VY} = \frac{e^2_\sigma}{2} \left|\frac{N}{2\pi}
\ln\frac{\sqrt{2}\, \sigma}{\Lambda}\right|^2
+\frac{N}{4\pi} u\,2|\sigma|^2\,.
\label{fdop4}
\eeq
The kinetic term (that's where $e^2_\sigma$ comes from) was not determined; however, we can take
it in the form obtained in the large-$N$ expansion, see (\ref{fdop1}),
since it is scale invariant and, hence,  does not violate Ward identities.

Combining 
\beq
e^2_\sigma =\frac{4\pi}{N}\,2|\sigma|^2
\label{fdop5} 
\eeq
(see \cite{SYhet}) with (\ref{fdop4}) we arrive at
\beqn
{\mathcal L}_{VY} 
&=&\frac{N}{4\pi}\,\frac{1}{2|\sigma|^2}\,| \,  \partial_\mu\sigma \,|^2+
 \frac{N}{4\pi}
\left\{ 2\cdot 2|\sigma|^2
 \left|
\ln\frac{\sqrt{2}\, \sigma}{\Lambda}\right|^2
+2|\sigma|^2\, u\right\}
\nonumber\\[3mm]
&=& \frac{N}{4\pi}
\left\{
\frac{1}{\,2\, |\cs|^2}\,\,
 | \partial_\mu\cs \, |^2
 +\Lambda^2
 \left[2|\cs |^2\,\left|\ln \cs\right|^2 + |\cs|^2\,u
 \right]
 \right\}.
 \label{fdop6}
\eeqn
It is obvious that the potential in (\ref{fdop3})
is drastically different from that in (\ref{fdop6}). 
For instance, (\ref{fdop3}) contains a single log, while (\ref{fdop6}) has the square of this logarithm.
We will comment on the difference and the reasons for its appearence \cite{DADA}
later. Now, let us have a closer look at the minima of
(\ref{fdop3}) and (\ref{fdop6}). The variable $\cs$ is complex, and there iare $N$ solutions
which differ by the phase,
\beq
\cs_* = \left|\cs_* \right|\exp\left(\frac{2\pi k}{N}\right)\,,\qquad k = 0,1, ..., N-1\,,
\eeq
$N$ equivalent vacua. This feature is obvious, and we will omit the phase setting $k=0$.
Thus, we focus on a real solution.
The minimum of (\ref{fdop3}) lies at
 \beq
 \cs_* = e^{-u/2}
 \label{wdop1}
 \eeq
while the corresponding value of $V_{\rm eff}$ is
\beq
V_{\rm eff} (\cs_* ) = \frac{N}{4\pi}\,\Lambda^2 \left(1- e^{-u}
\right)  .
 \label{wdop2}
\eeq
At the same time, the minimum of (\ref{fdop6}) lies at
\beq
\cs_* = \exp\left(-\frac{1}{2} + \sqrt{\frac{1}{4}-\frac{u}{2}}\right) =  e^{-u/2}\left(1-\frac{u^2}{4} + ...
\right)
\label{wdop3}
\eeq
implying that 
\beqn
V_{\rm VY} (\cs_* )
&=&  \frac{N}{4\pi}\,\Lambda^2 \,\left({1} - \sqrt{{1}-2 {u}} \right)\exp\left(-{1} + \sqrt{{1}-2 {u}}\right)
\nonumber\\[3mm]
& =&
  \frac{N}{4\pi}\,\Lambda^2 \left(1- e^{-u}
\right) \left(1-\frac{u^2}{6}  + ...
\right) .
\label{wdop4}
\eeqn 
The $\sigma$ masses are
\beq
m_\sigma^2 = \left\{
\begin{array}{l}
4\Lambda^2\,e^{-u}\,(1-u)\,,\\[2mm]
4\Lambda^2\,e^{-u}\,(1-u)\,(1-u^2+...)\,,
\end{array}
\right.
\label{wdop4}
\eeq
for (\ref{fdop3}) and (\ref{fdop6}) respectively.
The positions of the minima, the $\sigma$ masses
as well as the vacuum energy densities in these two cases
differ by $O(u^2)$ in relative units. They coincide in the leading and next-to-leading orders in $u$, however.

There are two questions to be discussed: (i) why the effective Lagrangians  (\ref{fdop3}) and (\ref{fdop6}),
 being essentially different, predict identical vacuum parameters in the leading and next-to-leading order in $u$;
 and (ii) why the  parameters extracted from the $1/N$ and Veneziano--Yankielowicz Lagrangians diverge 
 from each other at $O(u^2)$ and  higher orders.

The answer to the first question can be found in \cite{AdDVecSal}. While  the $1/N$ 
Lagrangian is defined unambiguously, the Veneziano--Yankielowicz method determines only the
superpotential part of the action. The kinetic part remains ambiguous. We got used to the fact that
variations of the kinetic part affect only terms with derivatives, which are totally irrelevant for the
potential part. This is not the case in supersymmtry.
The correct statement is that variations of the kinetic part term, in addition to derivative terms,
contains terms with $F\bar F$, which vanish in the vacuum ($F=0$) but alter the form of the potential outside
the vacuum points (minima of the potential). The only requirement to the kinetic term is that
it should obey all Ward identities (including anomalous) of the underlying microscopic theory.
For instance, in the case at hand, the simplest choice $\ln\bar\Sigma\, \ln \Sigma $
does the job. However, 
$$
\ln\bar\Sigma\, \ln \Sigma \left[1+ \frac{(\bar{D}^2 \ln\bar\Sigma)\, ({D}^2\ln \Sigma )}{\bar\Sigma\Sigma }
\right]
$$
does the job as well. In this latter case there is an additional factor
$$
\left[1 +\bar FF/(\bar\sigma^2\sigma^2)   +...  \right] $$
which reduces to 1 in the points where $F=0$ and changes the expression for $F$ (and, hence, the scalar potential)
outside minima (i.e. at $F\neq 0$).

The answer to the second question is even more evident.
The Veneziano--Yankielowicz Lagrangian (\ref{fdop6}) reflects the Ward identites
of the unperturbed CP$(N-1)$ model. That's the reason why the predictions following from this
Lagrangian fail at the level $O(u^2)$, but are valid at the level $O(u)$. 
We remind the reader that it was shown in \cite{SY1} that the vacuum energy density
at the level $O(u)$ is determined by the bifermion condensate in the 
conventional (unperturbed) CP$(N-1)$ model.

One last remark is in order here.
The kinetic term (\ref{fdop1}) is not canonic and singular at $\sigma=0$, implying that this point should be analyzed separately.
One can readily cast (\ref{fdop1}) in the canonic form by a change of variables. Upon this transformation 
$\sigma\to \tilde 
\sigma =2 \ln\sqrt{2}\sigma/\Lambda$ (assuming for simplicity $\sigma$ to be real and positivde),
the transformed potential (\ref{fdop2})    develops an extremum at 
$\sigma = 0$ (i.e. $\tilde\sigma \to -\infty$). This extremum is maximum rather than minimum.
Indeed, at $u=0$
\beq
\tilde{\mathcal L}_{\rm pot} = \frac{N\Lambda^2}{4\pi^2}\,(\tilde\sigma - 1)\,e^{\tilde\sigma}
+{\rm const}.
\label{mdop7}
\eeq
It is curious to note that (\ref{mdop7}) exactly coincides with the (two-dimensional)
dilaton effective Lagrangian derived in \cite{SMMS} on the basis of the most general
(anomalous) scale Ward identities.

\small
\begin{thebibliography}{99}
\itemsep -2pt

\bibitem{AdDVecSal}
  A.~D'Adda, A.~C.~Davis, P.~Di Vecchia and P.~Salomonson,
  %``An Effective Action For The Supersymmetric Cp**(N-1) Model,''
  Nucl.\ Phys.\  B {\bf 222}, 45 (1983).
  %%CITATION = NUPHA,B222,45;%%
  
  \bibitem{SMMS}
  A.~Migdal and M.~Shifman,
  %``Dilaton Effective Lagrangian In Gluodynamics,''
  Phys.\ Lett.\  B {\bf 114}, 445 (1982).
  %%CITATION = PHLTA,B114,445;%%

\end{thebibliography}

\end{document}
