\documentclass[epsfig,12pt]{article}
\usepackage{epsfig}
\usepackage{graphicx}

%%%%%%%%%
\usepackage{latexsym}
\usepackage{amsmath}
\usepackage{amssymb}
\usepackage{relsize}
\usepackage{geometry}
\geometry{letterpaper}
\usepackage{showlabels}
%%%%%%%%%%%%

\def\beq{\begin{equation}}
\def\eeq{\end{equation}}
\def\beqn{\begin{eqnarray}}
\def\eeqn{\end{eqnarray}}
\def\Tr{{\rm Tr}}
\newcommand{\nfour}{${\cal N}=4\;$}
\newcommand{\ntwo}{${\mathcal N}=2\,$}
\newcommand{\none}{${\mathcal N}=1\,$}
\newcommand{\ntt}{${\mathcal N}=(2,2)\,$}
\newcommand{\nzt}{${\mathcal N}=(0,2)\,$}
\newcommand{\cpn}{CP$(N-1)\,$}
\newcommand{\ca}{{\mathcal A}}
\newcommand{\cell}{{\mathcal L}}
\newcommand{\cw}{{\mathcal W}}
\newcommand{\vp}{\varphi}
\newcommand{\pt}{\partial}
\newcommand{\ve}{\varepsilon}
\newcommand{\gs}{g^{2}}
\newcommand{\zn}{$Z_N$}
\newcommand{\cd}{${\mathcal D}$}
\newcommand{\cde}{{\mathcal D}}
\newcommand{\cf}{${\mathcal F}$}
\newcommand{\cfe}{{\mathcal F}}

\newcommand{\gsim}{\lower.7ex\hbox{$
\;\stackrel{\textstyle>}{\sim}\;$}}
\newcommand{\lsim}{\lower.7ex\hbox{$
\;\stackrel{\textstyle<}{\sim}\;$}}

\renewcommand{\theequation}{\thesection.\arabic{equation}}

%%%%%%%%%%%%%
%%%%%%%%%%%
%% common definitions
\def\stackunder#1#2{\mathrel{\mathop{#2}\limits_{#1}}}
\def\beqn{\begin{eqnarray}}
\def\eeqn{\end{eqnarray}}
\def\nn{\nonumber}
\def\baselinestretch{1.1}
\def\beq{\begin{equation}}
\def\eeq{\end{equation}}
\def\ba{\beq\new\begin{array}{c}}
\def\ea{\end{array}\eeq}
\def\be{\ba}
\def\ee{\ea}
\def\stackreb#1#2{\mathrel{\mathop{#2}\limits_{#1}}}
\def\Tr{{\rm Tr}}
%\newcommand{\gsim}{\lower.7ex\hbox{$\;\stackrel{\textstyle>}{\sim}\;$}}
% \newcommand{\lsim}{\lower.7ex\hbox{$
%\;\stackrel{\textstyle<}{\sim}\;$}}
%\newcommand{\nfour}{${\mathcal N}=4$ }
%\newcommand{\ntwo}{${\mathcal N}=2$ }
\newcommand{\ntwon}{${\mathcal N}=2$}
\newcommand{\ntwot}{${\mathcal N}= \left(2,2\right) $ }
\newcommand{\ntwoo}{${\mathcal N}= \left(0,2\right) $ }
%\newcommand{\none}{${\mathcal N}=1$ }
\newcommand{\nonen}{${\mathcal N}=1$}
%\newcommand{\vp}{\varphi}
%\newcommand{\pt}{\partial}
%\newcommand{\ve}{\varepsilon}
%\newcommand{\gs}{g^{2}}
%\newcommand{\qt}{\tilde q}
\renewcommand{\theequation}{\thesection.\arabic{equation}}

%%
\newcommand{\p}{\partial}
\newcommand{\wt}{\widetilde}
\newcommand{\ov}{\overline}
\newcommand{\mc}[1]{\mathcal{#1}}
\newcommand{\md}{\mathcal{D}}

\newcommand{\GeV}{{\rm GeV}}
\newcommand{\eV}{{\rm eV}}
\newcommand{\Heff}{{\mathcal{H}_{\rm eff}}}
\newcommand{\Leff}{{\mathcal{L}_{\rm eff}}}
\newcommand{\el}{{\rm EM}}
\newcommand{\uflavor}{\mathbf{1}_{\rm flavor}}
\newcommand{\lgr}{\left\lgroup}
\newcommand{\rgr}{\right\rgroup}

\newcommand{\Mpl}{M_{\rm Pl}}
\newcommand{\suc}{{{\rm SU}_{\rm C}(3)}}
\newcommand{\sul}{{{\rm SU}_{\rm L}(2)}}
\newcommand{\sutw}{{\rm SU}(2)}
\newcommand{\suth}{{\rm SU}(3)}
\newcommand{\ue}{{\rm U}(1)}
%%%%%%%%%%%%%%%%%%%%%%%%%%%%%%%%%%%%%%%
%  Slash character...
\def\slashed#1{\setbox0=\hbox{$#1$}             % set a box for #1
   \dimen0=\wd0                                 % and get its size
   \setbox1=\hbox{/} \dimen1=\wd1               % get size of /
   \ifdim\dimen0>\dimen1                        % #1 is bigger
      \rlap{\hbox to \dimen0{\hfil/\hfil}}      % so center / in box
      #1                                        % and print #1
   \else                                        % / is bigger
      \rlap{\hbox to \dimen1{\hfil$#1$\hfil}}   % so center #1
      /                                         % and print /
   \fi}                                        %

%%EXAMPLE:  $\slashed{E}$ or $\slashed{E}_{t}$

%%

\newcommand{\LN}{\Lambda_\text{SU($N$)}}
\newcommand{\sunu}{{\rm SU($N$) $\times$ U(1) }}
\newcommand{\sunun}{{\rm SU($N$) $\times$ U(1)}}
\def\cfl {$\text{SU($N$)}_{\rm C+F}$ }
\def\cfln {$\text{SU($N$)}_{\rm C+F}$}
\newcommand{\mUp}{m_{\rm U(1)}^{+}}
\newcommand{\mUm}{m_{\rm U(1)}^{-}}
\newcommand{\mNp}{m_\text{SU($N$)}^{+}}
\newcommand{\mNm}{m_\text{SU($N$)}^{-}}
\newcommand{\AU}{\mc{A}^{\rm U(1)}}
\newcommand{\AN}{\mc{A}^\text{SU($N$)}}
\newcommand{\aU}{a^{\rm U(1)}}
\newcommand{\aN}{a^\text{SU($N$)}}
\newcommand{\baU}{\ov{a}{}^{\rm U(1)}}
\newcommand{\baN}{\ov{a}{}^\text{SU($N$)}}
\newcommand{\lU}{\lambda^{\rm U(1)}}
\newcommand{\lN}{\lambda^\text{SU($N$)}}
%\newcommand{\Tr}{{\rm Tr\,}}
\newcommand{\bxir}{\ov{\xi}{}_R}
\newcommand{\bxil}{\ov{\xi}{}_L}
\newcommand{\xir}{\xi_R}
\newcommand{\xil}{\xi_L}
\newcommand{\bzl}{\ov{\zeta}{}_L}
\newcommand{\bzr}{\ov{\zeta}{}_R}
\newcommand{\zr}{\zeta_R}
\newcommand{\zl}{\zeta_L}
\newcommand{\nbar}{\ov{n}}

\newcommand{\CPC}{CP($N-1$)$\times$C }
\newcommand{\CPCn}{CP($N-1$)$\times$C}

\newcommand{\lar}{\lambda_R}
\newcommand{\lal}{\lambda_L}
\newcommand{\larl}{\lambda_{R,L}}
\newcommand{\lalr}{\lambda_{L,R}}
\newcommand{\blar}{\ov{\lambda}{}_R}
\newcommand{\blal}{\ov{\lambda}{}_L}
\newcommand{\blarl}{\ov{\lambda}{}_{R,L}}
\newcommand{\blalr}{\ov{\lambda}{}_{L,R}}

\newcommand{\bgamma}{\ov{\gamma}}
\newcommand{\bpsi}{\ov{\psi}{}}
\newcommand{\bphi}{\ov{\phi}{}}
\newcommand{\bxi}{\ov{\xi}{}}

\newcommand{\ff}{\mc{F}}
\newcommand{\bff}{\ov{\mc{F}}}

\newcommand{\eer}{\epsilon_R}
\newcommand{\eel}{\epsilon_L}
\newcommand{\eerl}{\epsilon_{R,L}}
\newcommand{\eelr}{\epsilon_{L,R}}
\newcommand{\beer}{\ov{\epsilon}{}_R}
\newcommand{\beel}{\ov{\epsilon}{}_L}
\newcommand{\beerl}{\ov{\epsilon}{}_{R,L}}
\newcommand{\beelr}{\ov{\epsilon}{}_{L,R}}

\newcommand{\bi}{{\bar \imath}}
\newcommand{\bj}{{\bar \jmath}}
\newcommand{\bk}{{\bar k}}
\newcommand{\bl}{{\bar l}}
\newcommand{\bm}{{\bar m}}

\newcommand{\nz}{{n^{(0)}}}
\newcommand{\no}{{n^{(1)}}}
\newcommand{\bnz}{{\ov{n}{}^{(0)}}}
\newcommand{\bno}{{\ov{n}{}^{(1)}}}
\newcommand{\Dz}{{D^{(0)}}}
\newcommand{\Do}{{D^{(1)}}}
\newcommand{\bDz}{{\ov{D}{}^{(0)}}}
\newcommand{\bDo}{{\ov{D}{}^{(1)}}}
\newcommand{\sigz}{{\sigma^{(0)}}}
\newcommand{\sigo}{{\sigma^{(1)}}}
\newcommand{\bsigz}{{\ov{\sigma}{}^{(0)}}}
\newcommand{\bsigo}{{\ov{\sigma}{}^{(1)}}}

\newcommand{\rrenz}{{r_\text{ren}^{(0)}}}
%%%%%%%%%%%%%%%%%%%%%%%

\begin{document}

%%%%%%%%%%%%%%%%%%%%%%%%%%%%%%%%


\begin{titlepage}

\begin{flushright}
FTPI-MINN-09/??, UMN-TH-????/09\\
October 21/DRAFT
\end{flushright}

\vspace{1cm}

\begin{center}
{  \Large \bf  Large-\boldmath{$N$} Solution of the Heterotic\\[2mm]
 CP\boldmath{$(N-1)$} Model with Twisted Masses}
\end{center}



\vspace{1mm}

\begin{center}

 {\large
 \bf   P. Bolokhov,  M.~Shifman$^{\,a}$ and \bf A.~Yung$^{\,\,a,b}$}
\end {center}

\begin{center}

%\vspace{3mm}

$^a${\it  William I. Fine Theoretical Physics Institute,
University of Minnesota,
Minneapolis, MN 55455, USA}\\
%[1mm]
$^{b}${\it Petersburg Nuclear Physics Institute, Gatchina, St. Petersburg
188300, Russia
}
\end{center}


\begin{center}
{\large\bf Abstract}
\end{center}
We address a number of unanswered questions on the \nzt deformation of the \cpn model with
twisted masses.  In particular, we
complete the program of solving \cpn model with twisted masses in the large-$N$
limit. In hep-th/0512153 nonsupersymmetric version of the model with the $Z_N$
symmetric twisted masses was analyzed in the framework of Witten's method. In arXiv:0803.0698 
this analysis was extended: we presented the 
large-$N$ solution of the heterotic \nzt \cpn model with no twisted masses.
Here we solve the heterotic \nzt \mbox{\cpn}  model with nonvanishing twisted
masses. Dynamical scenarios at large and small $m$ are studied
($m$ is the twisted mass scale).
In a certain limiting case we discuss the mirror representation for the heterotic model.

\vspace{2cm}




\end{titlepage}

\newpage

%\tableofcontents

\section{Introduction}

Two-dimensional  CP$(N-1)$ models with twisted masses
emerged as effective low-energy theories on
the worldsheet of non-Abelian strings in a class of 
four-dimensional \ntwo\, gauge theories with unequal (s)quark 
masses~\cite{ABEKY,SYmon,HT2} (for reviews see  \cite{Trev}). 
Deforming these models in various ways (i.e. breaking supersymmetry down to \none and to nothing)
one arrives at nonsupersymmetric or heterotic \cpn models.\footnote{
Strictly speaking, the full derivation of the
heterotic \cpn model with twisted masses, valid for arbitrary values of
the deformation parameters,  from the microscopic bulk theory, is still absent.
However, at small values of the deformation parameters,
such a derivation is quite straightforward.} These two-dimensional models are very 
interesting on their own,
since they exhibit nontrivial dynamics with or without phase transitions 
as one varies the twisted mass scale.
In this paper we will present the large-$N$ solution of the \nzt \mbox{\cpn}  model with twisted
masses. As a warm up exercise we analyze this model in the limit of vanishing heterotic deformation,
i.e. the \ntt \cpn model with twisted masses (at $N\to\infty$). Both solutions 
that we report here are based on the method developed by Witten
\cite{W79,W93} (see also \cite{dadvl}) and extended in \cite{SYhet} to include the heterotic deformation.
For certain purposes we find it convenient to invoke the mirror
representation \cite{MR1,MR2}. An \nzt \cpn$\times C$
model on the string world sheet in the bulk theory deformed by $\mu\ca^2$
was suggested by Edalati and Tong \cite{EdTo}.
It was derived from the bulk theory in \cite{SY1} (see also \cite{BSY1,BSY2}).
Finally, the heterotic \nzt  \cpn model with twisted masses was formulated in \cite{BSY3}.

\section{Generalities}

\ntt supersymmtric \cpn sigma model was originally constructed 
\cite{orco}
in terms of \none superfields. Somewhat later it was realized \cite{Bruno}
that 
\none supersymmetry is automatically elevated up to \ntwo
provided the target manifold of the sigma model in question is K\"ahlerian (for reviews see \cite{rev1,rev2}).
The Witten index \cite{WI} of the \cpn model is $N$, implying unbroken supersymmetry
and $N$ degenerate vacua at zero energy density.
The \cpn manifold is compact; therefore, superpotential is impossible.
One can introduce mass terms, however, through the twisted masses \cite{twisted}.
The model is asymptotically free \cite{BelPo}, a dynamical scale $\Lambda$ is generated through dimensional transmutation. If the scale of the twisted masses is much larger than $\Lambda$, the theory is at weak coupling.
Otherwise it is at strong coupling. A priori, there are $N$ distinct twisted mass parameters.
However, in the absence of the heterotic deformation one of them is unobservable
(see below). In this case the model is characterized by the coupling constant $g^2$,
the vacuum angle $\theta$ and the twisted mass parameters $m_1,\,m_2,\, ..., m_N$
with the constraint
\beq
m_1 + m_2 + ... +m_N =0\,.
\label{one}
\eeq
Introducing a heterotic deformation, generally speaking, we eliminate the above constraint.
The twisted masses are arbitrary complex parameters. Of special interest 
in some instances (for example, in studying possible phase transitions)
is the $Z_N$ symmetric choice
\beq
m_k = m\exp\left(\frac{2\pi \,i\, k}{N}\right)\,,\qquad k = 0,1,2, ... ,N-1\,.
\label{two}
\eeq
The set (\ref{two}) will be referred to as the $Z_N$-symmetric masses.
Then the constraint
(\ref{one}) is automatically satisfied. Without loss of generality $m$ can be assumed
to be real and positive. 

Where necessary, we mark the bare coupling constant by the subscript 0 and introduce
the inverse parameter $\beta$ as follows:
\beq
\beta = \frac{1}{g_0^2}\,.
\eeq
At large $N$, in the 't Hooft limit,  the parameter $\beta$ scales as $N$.

There are two equivalent languages commonly used in description of the \cpn model:
the geometric language ascending to \cite{Bruno} (see also \cite{rev2}), and the
so-called gauged formulation ascending to \cite{W79,W93}. Both have their convenient and less convenient sides.
We will discuss both formulations although construction of the $1/N$ expansion is
more convenient within the framework of the gauged formulation. At $|m|/\Lambda \to 0$
the elementary fields of the
gauged formulation (they form an  $N$-plet) are in one-to-one correspondence with the kinks in the geometric formulation.
The multiplicity of kinks -- the fact they they enter in $N$-plets --
can be readily established \cite{adam} using the mirror representation \cite{MR1}. We will discuss this in more detail
later. 

\section{The model}

\subsection{Gauged formulation, no heterotic deformation}

In this section we will briefly review the gauged formulation \cite{W79,W93} of
the \mbox{\ntt} \cpn model with twisted masses
\cite{twisted}, i.e. we set the heterotic deformation coupling
$\gamma = 0$.
This formulation is built on an $N$-plet of complex scalar fields $n^i$ where $i=1,2,...,N$.
We impose the constraint
\beq
\bar n_i \,n^i = 2\beta \,.
\eeq
This leaves us with $2N-1$ real bosonic degrees of freedom. To eliminate one extra degree
of freedom we impose a local U(1) invariance $n^i(x)\to e^{i\alpha(x)} n^i(x)$.
To this end we introduce a gauge field $A_\mu$ which converts the partial derivative into the
covariant one,
\beq
\partial_\mu\to \nabla_\mu \equiv \partial_\mu -i\,  A_\mu\,.
\eeq
The field $A_\mu$ is auxiliary; it enters in the Lagrangian without derivatives. The kinetic term of the
$n$ fields is
\beq
\cell  = \left|\nabla_\mu n^i\right|^2\,.
\eeq
The superpartner to the field $n^i$ is an $N$-plet of complex two-component spinor fields $\xi^i$,
\beq
\xi^i =\left\{\begin{array}{l}
\xi^i_R\\[2mm]
\xi^i_L
\end{array}
\right.\,,
\eeq
subject to the constraint
\beq
\bar{n}_i\,\xi^i =0\,,\qquad \bar\xi_i\,n^i = 0\,.
\label{npxi}
\eeq
Needless to say, the auxiliary field $A_\mu$ has a complex scalar superpartner $\sigma$ 
and a two-component complex spinor superpartner $\lambda$; both enter without derivatives.
The full \ntt-symmetric Lagrangian is\,\footnote{This is, obviously, the Euclidean version.}
\beqn
\cell &=& 
\frac{1}{e_0^2}\left(\frac{1}{4} F_{\mu\nu}^2 +\left|\pt_\mu\sigma\right|^2 + \frac{1}{2}D^2
+\bar\lambda \, i\bar{\sigma}^\mu\pt_\mu\,\lambda
\right) + i\,D\left(\bar{n}_i n^i -2\beta
\right)
\nonumber\\[3mm]
&+&
\left|\nabla_\mu n^i\right|^2+ \bar{\xi}_i\, i\bar{\sigma}^\mu\nabla_\mu\,\xi^i
+ 2\sum_i\left|\sigma-\frac{m_i}{\sqrt 2}\right|^2\, |n^i|^2
\nonumber\\[3mm]
&+&
i\sqrt{2}\,\sum_i \left( \sigma -\frac{m_i}{\sqrt 2}\right)\bar\xi_{Ri}\, \xi^i_L +i\sqrt{2}\,\bar{n}_i
\left(\lambda_R\xi^i_L - \lambda_L\xi^i_R
\right)
\nonumber\\
&+&
i\sqrt{2}\,\sum_i \left( \bar\sigma -\frac{\bar{m}_i}{\sqrt 2}\right)\bar\xi_{Li}\, \xi^i_R +i\sqrt{2}\,{n}^i
\left(\lambda_L\bar\xi_{Ri} - \lambda_R\bar\xi_{Li}
\right),
\label{bee31}
\eeqn
where $m_i$ are twisted mass parameters, and the limit $e_0^2\to\infty$ is implied. 
\marginpar{\tiny Znak poslednih chlenov v dvuh poslednih
strochkah ne sovpapadet s Pashey!}
Moreover,
\beq
\bar\sigma^\mu = \{1,\,i\sigma_3\}\,,
\eeq
see Appendix A.

It is clearly seen that the auxiliary field $\sigma$
enters in (\ref{bee31}) only through the  combination
\beq
\sigma -\frac{m_i}{\sqrt 2}\,.
\label{combi}
\eeq
By an appropriate shift of $\sigma$
one can always redefine the twisted mass parameters in such a way that the constraint
(\ref{one}) is satisfied.
 The U(1) gauge symmetry is built in. This symmetry eliminates one bosonic degree of freedom, leaving us with $2N-2$ dynamical bosonic degrees of freedom inherent to CP$(N-1)$ model.



\subsection{Geometric formulation, \boldmath{$\gamma=0$} }

Here we will  briefly review the \ntt supersymmetric \cpn models in the
geometric formulation. 
The target space is the $N-1$-dimensional K\"ahler manifold 
parametrized by the fields $\phi^{i},\,\phi^{\dagger\,\bar j}$, $\,i,\bar j=1,\ldots,N-1$,
which are the lowest components of the chiral and antichiral superfields 
\beq
\Phi^{i}(x^{\mu}+i\bar \theta \gamma^{\mu} \theta),\qquad \bar\Phi^{\bar j}(x^{\mu}-i\bar \theta \gamma^{\mu} \theta)\,,
\eeq
where
\beqn
&&
x^{\mu}=\{t,z\},\qquad \bar \theta=\theta^{\dagger}\gamma^{0},\qquad \bar \psi=\psi^{\dagger}\gamma^{0}
\nonumber
\\[2mm]
&&\gamma^{0}=\gamma^t=\sigma_2\,,\qquad \gamma^{1}=\gamma^z = i\sigma_1\,,\qquad \gamma_{5} 
\equiv\gamma^0\gamma^1 = \sigma_3\,.
\eeqn
With no twisted mass the Lagrangian  is \cite{Bruno}
(see also \cite{WessBagger})
\begin{equation}
\label{eq:kinetic}
{\cell}_{m=0}= \int d^{4 }\theta K(\Phi, \bar\Phi)
=G_{i\bar j} \left[\partial^\mu \bar\phi^{\,\bar j}\, \partial_\mu\phi^{i}
+i\bar \psi^{\bar j} \gamma^{\mu} \cde_{\mu}\psi^{i}\right]
-\frac{1}{2}\,R_{i\bar jk\bar l}\,(\bar\psi^{\bar j}\psi^{i})(\bar\psi^{\bar l}\psi^{k}).
\end{equation}
Here $K(\Phi, \bar\Phi)$ is the K\"ahler potential, while
$$
G_{i\bar j}=\frac{\partial^{2} K(\phi,\,\bar\phi)}{\partial \phi^{i}\partial \bar\phi^{\,\bar j}}
$$
 is the K\"ahler metric,
$R_{i\bar jk\bar l}$ is the Riemann tensor \cite{Helgason}, 
\beq
R_{i\bar{j} k\bar{m}} = - \frac{g_0^2}{2}\left(G_{i\bar{j}}G_{k\bar{m}} +
G_{i\bar{m}}G_{k\bar{j}}
\right)\,.
\label{640}
\eeq
Moreover,
$$ \cde_{\mu}\psi^{i}=
\partial_{\mu}\psi^{i}+\Gamma^{i}_{kl}\partial_{\mu} \phi^{k}\psi^{l}
$$
is the covariant derivative.
The Ricci tensor $R_{i\bar j}$ is proportional to the metric \cite{Helgason},
\beq
\label{eq:RG}
R_{i\bar{j}} = \frac{g_{0}^2}{2}\,  N \, G_{i\bar{j}}\,.
\eeq
For the massless CP($N\!-\!1)$ model 
a particular choice of the K\"ahler potential
\begin{equation}
\label{eq:kahler}
K_{m=0}=\frac{2}{g_{0}^{2}}\ln\left(1+\sum_{i,\bar j=1}^{ N-1}\bar\Phi^{\,\bar j}\delta_{\bar j i}\Phi^{i}\right)
\end{equation}
corresponds to the round Fubini--Study metric.

Let us  briefly remind how one can introduce the twisted mass parameters \cite{twisted, Dor}.
The theory (\ref{eq:kinetic}) can be interpreted as an ${\mathcal N}=1$ theory of $d$ chiral superfields 
in four dimensions.  The theory possesses some number $r$ of U(1) isometries 
parametrized by $t^{a}$, $a=1,\ldots,r$.
The Killing vectors of the isometries can be expressed via derivatives of the Killing 
potentials $D^{a}(\phi, \phi^{\dagger})$,
\begin{equation}
\label{eq:KillD}
\frac{{d}\phi^{i}}{{  d}\,t_{a}}=-iG^{i\bar j}\,\frac{\partial D^{a}}{\partial \bar\phi^{ \,\bar j}}
\,,\qquad 
\frac{{d}\bar\phi^{ \,\bar j}}{{  d}\,t_{a}}=iG^{i\bar j}\,\frac{\partial D^{a}}{\partial \phi^{i}}\,.
\end{equation}
This defines U(1) Killing potentials up to additive constants.

In the case of  CP$(N\!-\!1)$ there are $N\!-\!1$ isometries 
 evident from the expression (\ref{eq:kahler}) for the K\"ahler potential, 
\begin{equation}
\label{eq:iso}
\delta\phi^{i}=-i\delta t_{a} (T^{a})^{i}_{k}(\phi)^{k}\,,\qquad 
\delta\bar\phi^{\,\bar j}=i\delta t_{a}(T^{a})^{\bar j}_{\bar l}\bar\phi^{\,\bar l}\,,
\qquad a=1,\ldots, N-1\,,
\end{equation}
(together with the similar variation of fermionic fields),
where the  generators $T^{a}$ have a simple diagonal form,
\begin{equation}
(T^{a})^{i}_{k}=\delta^{i}_{a}\delta^{a}_{k}\,, \qquad a=1,\ldots,N-1\,.
\end{equation}
 The explicit form of the Killing potentials $D^{a}$ in CP$(N\!-\!1)$ with the Fubini--Study metric is
\beq
\label{eq:KillF}
D^{a}=\frac{2}{g_{0}^{2}}\,\frac{\bar\phi\, T^{a}\phi}{1+\bar\phi\,\phi}\,,
\qquad a=1,\ldots,N-1\,.
\eeq
Here we use the matrix notation implying that $\phi$ is a column $\phi^{i}$ and 
$\bar\phi$ is a row $\bar\phi^{ \bar j}$.

The isometries allow us  to introduce an interaction with $r$ {\em external} 
U(1) gauge 
superfields $V_{a}$ by modifying, in a gauge invariant way,  the K\"ahler potential (\ref{eq:kahler}),
\begin{equation}
\label{eq:mkahler}
K_{m=0}(\Phi, \bar\Phi)\to
K_{m}(\Phi, \bar\Phi,V)\,.
\end{equation}
For CP$(N\!-\!1)$ this modification takes the form
\begin{equation}
\label{eq:mkahlerp}
K_{m}=\frac{2}{g_{0}^{2}}\ln \left(1+\bar\Phi\,{\rm e}^{V_{a}T^{a}}\Phi\right)\,.
\end{equation}
In every gauge multiplet $V_{a}$ let us retain only the $A^{a}_{x}$ and $A^{a}_{y}$ 
components of the gauge potentials taking them to be just constants,
\beq
V_{a}=-m_{a}\bar \theta(1+\gamma_{5})\theta -\bar m_{a}\bar \theta(1-\gamma_{5})\theta\,,
\end{equation}
where we introduced complex masses  $m_{a}$ as linear combinations of 
constant U(1) gauge potentials,
\beq
m_{a}=A^{a}_{y}+iA^{a}_{x}\,,\qquad \bar m_{a}=m_{a}^{*}=A^{a}_{y}-iA^{a}_{x}\,.
\end{equation}
In spite  of the explicit $\theta$ dependence the introduction of masses does not 
break ${\cal N}=2$ supersymmetry.  One way to see this is to notice that the mass parameters 
can be viewed as the lowest components of the twisted chiral superfields
$D_{2}\bar D_{1}V_{a}$.

Now we can go back to two dimensions implying that there is no dependence
on $x$ and $y$ in the chiral fields.  It gives us the Lagrangian with the twisted masses 
included \cite{twisted, Dor}:
\beqn
\label{eq:mtwist}
{\mathcal L}_{m}
&=&
 \int d^{4 }\theta \,K_{m}(\Phi, \bar\Phi,\,V)
=G_{i\bar j}\, g_{MN}\left[ {\cde}^M \bar\phi^{\,\bar j}\, {\cde}^{N} \phi^{i}
+i\,\bar \psi^{\bar j} \gamma^{M}\,D^{N} \psi^{i}\right]
\nonumber\\[3mm]
&-&
\frac{1}{2}\,R_{i\bar jk\bar l}\,(\bar\psi^{\bar j}\psi^{i})(\bar\psi^{\bar l}\psi^{k})\,,
\eeqn
where $G_{i\bar j} =\partial_{i}\partial_{\bar j}K_{m}|_{\theta=\bar\theta=0}$ is the K\"ahler metric 
and summation over $M$ includes, besides $M=\mu=0,1$, also 
$M=+,-$. 
The  metric $g_{MN}$ and extra gamma-matrices are
\begin{equation}
\label{eq:metric}
g_{MN}=\left(\begin{array}{crrr}1& 0& 0 & 0 \\0 & -1 & 0 & 0 \\[1mm]0 & 0 & 0 & -\frac 1 2 \\[1mm]0 & 0 & -\frac 1 2 & 0\end{array}\right),\qquad
\gamma^{+}=-i(1+\gamma_{5})\,,\quad
\gamma^{-}=i(1-\gamma_{5})\,.
\end{equation}
The gamma-matrices satisfy the following algebra:
\beq
\bar\Gamma^{M}\Gamma^{N}+\bar\Gamma^{N}\Gamma^{M}=2 g^{MN}\,,
\eeq
where the set $\bar\Gamma^{M}$ differs from $\Gamma^{M}$  by interchanging of
the $+,-$ components, $\bar\Gamma^{\pm}=\Gamma^{\mp}$.
The gauge covariant derivatives ${\cal D}^M$ are defined as
\beqn
&&
{\mathcal D}^{\mu}\phi=\partial^{\mu}\phi\,,\qquad {\cal D}^{+}\phi=-\bar m_{a}T^{a}\phi\,,
\qquad  {\cal D}^{-}\phi=m_{a}T^{a}\phi\,,
\nonumber\\[1mm]
&& {\cal D}^{\mu}\bar\phi
=\partial^{\mu}\bar\phi\,,
\quad ~{\cde}^{+}\bar\phi =\bar\phi\, T^{a}\bar m_{a}\,,
\qquad  {\cde}^{-}\bar\phi =-\bar \phi\, T^{a} m_{a}\,,
\nonumber\\
\eeqn
and similarly for ${\cal D}^{M}\psi$, while the general covariant derivatives $D^{M}\psi$ are
\begin{equation}
D^{M}\psi^{i}=
{\cal D}^{M}\psi^{i}+\Gamma^{i}_{kl}\,{\cal D}^{M}\! \phi^{k}\,\psi^{l}\,.
\end{equation}



\subsection{Gauged formulation, switching on the heterotic deformation}
\label{gfsothd}

In order to deform the \cpn model breaking \ntt down to \mbox{\nzt}
we must introduce a right-handed spinor field $\zeta_R$ whose  target space is $C$ 
(with a bosonic superpartner \cf), which is coupled to other fields as follows
\cite{EdTo,SY1}:
\beqn
\Delta\cell 
&=&
\bar\zeta_R\, i\pt_L\, \zeta_R   +\bar\cfe\,\cfe
	\nonumber\\[2mm]
&-&
 2i\, \omega \, \bar\lambda_L\, \zeta_R -  2i\, \bar{\omega} \, \bar\zeta_R\, \lambda_L +
	 2i\, \omega \, \cfe\, \sigma +
 2i\, \bar{\omega} \, \bar\cfe\, \bar{\sigma}\,,
 \label{deforte}
\eeqn
where we define
\beq
\omega = \sqrt{2\beta}\,\delta\,.
\eeq
This term must be added to the \ntt Lagrangian (\ref{bee31}). 
It is quite obvious that the dependence on (\ref{combi}) is gone. 
The deformation term (\ref{deforte}) has a separate dependence on $\sigma$, not reducible to the
combination (\ref{combi}). Therefore, for a generic choice, all $N$ twisted mass
parameters $m_1,\,m_2,\, ..., m_N$ become observable, Eq. (\ref{one}) is no longer valid.

\vspace{1mm}

Eliminating \cf, $\bar\cfe$  and $\bar\lambda ,\,\lambda$ we get
\beq
\Delta\cell = 4\, |\omega |^2\,|\sigma |^2\,,
\eeq
while the constraints (\ref{npxi}) are replaced by
\beqn
\bar{n}_i\,\xi^i_L
&= &
0\,,\qquad \bar\xi_{Ri}\,n^i = 0\,,
\nonumber\\[2mm]
\bar{n}_i\,\xi^i_R
&= &
 -\sqrt{2}\,\omega\zeta_R\,,\qquad \bar\xi_{Ri}\,n^i =  -\sqrt{2}\,\bar\omega\bar\zeta_R\,.
\label{npxip}
\eeqn

We still have to discuss how the parameter $\omega$ is related to other
deformation parameters (which are equivalent to $\omega$), and their $N$ dependence.
We want to single out appropriate powers of $N$ so that the large-$N$ limit will be smooth.

\subsection{Geometric formulation, \boldmath{$\tilde\gamma \neq 0$}} 
\label{gftgnz}

The parameter of the heterotic deformation in the geometric formulation will be denoted by
$\tilde \gamma$ (the tilde appears here for historical reasons;
perhaps, in the future it will be reasonable to omit it).

To obtain the Lagrangian of the heterotically deformed model
we act as follows \cite{BSY3}: we start from (\ref{eq:kinetic}), add the
 right-handed spinor field $\zeta_R$, with the same kinetic term as in Sect.~\ref{gfsothd}, and
 add the bifermion terms
 \beq
 \frac{\tilde\gamma \,g_0}{\sqrt 2}
 \left[\zeta_RG_{i\bar j}\left(i\pt_L\bar\phi^{\,\bar j}
 \right)\psi_R^i + \bar\zeta_R G_{i\bar j}\left(i\pt_L\phi^{i}
 \right)\bar\psi_R^{\,\bar j} 
 \right].
 \eeq
Next, we change the four-fermion terms exactly in the same way this was done in
 \cite{SY1},
 namely
 \beqn
&-&\frac{1}{2} \,R_{i\bar jk\bar l}\left[\left(\bar\psi^{\bar j}\psi^{i}\right)\left(\bar\psi^{\bar l}\psi^{k}\right)
\left(\bar\psi^{\bar j}\psi^{i}\right)\left(\bar\psi^{\bar l}\psi^{k}\right)
\right]
\nonumber\\[3mm]
&\to&
- \frac{g_0^2}{2}\left( G_{i\bar j}\psi^{\dagger\, \bar j}_R\, \psi^{ i}_R\right)
\left( G_{k\bar m}\psi^{\dagger\, \bar m}_L\, \psi^{ k}_L\right)+
\frac{g_0^2}{2}\left(1-|\tilde\gamma|^2\right)
\left( G_{i\bar j}\psi^{\dagger\, \bar j}_R\, \psi^{ i}_L\right)
\left( G_{k\bar m}\psi^{\dagger\, \bar m}_L\, \psi^{ k}_R\right)\,,
\nonumber
\\[4mm]
&-&
 \frac{g_0^2}{2} \, |\tilde{\gamma} |^2 \,\left(\zeta_R^\dagger\, \zeta_R
\right)\left(G_{i\bar j}\,  \psi_L^{\dagger\,\bar j}\psi_L^i\right),
\eeqn
 where the first line represents the last term in Eq.~(\ref{eq:kinetic}), and we used the identity~(\ref{640}).
 
 
  If one of the twisted masses from the set  $\{m_1,\,m_2,\, ..., m_N\}$ vanishes,
then this is the end of the story. However, with more general twisted mass sets,
for instance, for the \zn-symmetric masses, one should take into account an extra contribution.
Occurrence of this contribution can be seen in a relatively concise
manner using the superfield formalism of \cite{SY1}. 
 
\beqn
L_{{\rm heterotic}} && 
= 
\zeta_R^\dagger \, i\partial_L \, \zeta_R  + 
\left[\gamma\, g_0^2 \, \zeta_R  \, G_{i\bar j}\,  \big( i\,\partial_{L}\phi^{\dagger\,\bar j} \big)\psi_R^i
+{\rm H.c.}\right]
\nonumber
\\[4mm]
&&
 -g_0^4\, |\gamma |^2 \,\left(\zeta_R^\dagger\, \zeta_R
\right)\left(G_{i\bar j}\,  \psi_L^{\dagger\,\bar j}\psi_L^i\right)
\nonumber
\\[4mm]
&&
+G_{i\bar j} \big[\partial_\mu \phi^{\dagger\,\bar j}\, \partial_\mu\phi^{i}
+i\bar \psi^{\bar j} \gamma^{\mu} D_{\mu}\psi^{i}\big]
\nonumber
\\[4mm]
&&
- \frac{g_0^2}{2}\left( G_{i\bar j}\psi^{\dagger\, \bar j}_R\, \psi^{ i}_R\right)
\left( G_{k\bar m}\psi^{\dagger\, \bar m}_L\, \psi^{ k}_L\right)
\nonumber
\\[4mm]
&&
+\frac{g_0^2}{2}\left(1-2g^2_0|\gamma|^2\right)
\left( G_{i\bar j}\psi^{\dagger\, \bar j}_R\, \psi^{ i}_L\right)
\left( G_{k\bar m}\psi^{\dagger\, \bar m}_L\, \psi^{ k}_R\right)\,,
\label{cpn-1g}
\eeqn
where we used (\ref{640}).

The relation between $\tilde\gamma$ and $\delta$
is as follows \cite{BSY3}:
\beq
\tilde\gamma = \sqrt{2} \frac{\delta}{\sqrt{1+ 2 |\delta |^2}},
\eeq
\marginpar{\tiny Discuss the phase}
implying that $\tilde \gamma$ does {\em not} scale with $N$ in the 't Hooft limit.


\section{Large-\boldmath{$N$} solution
of the \cpn model with twisted masses}

\section{Large-\boldmath{$N$} solution
of the  heterotic\\ \cpn model with twisted masses}

To build construct an effective action allowing us to explore the vacuum structure of the
model under consideration, we integrate over all but one given $ n^l $ filed (and its superpartner $ \xi^l $).
One can alway choose this fixed (unintegrated) field to be the last one, $n^N$.
To the leading order in $1/N$ we can (and will) ignore the gauge field $ A_\mu $.
The fields $ n^i $ and $ \xi^i $  ($ i = 1,...\, N-1 $) enter the Lagrangian quadratically,
\beqn
\cell &=&
 \nbar{}_i 
		   \left( - \p_k^2 ~+~ 2\,\Bigl| \sigma - \frac{m^i}{\sqrt{2}} \Bigr|^2 +  i\, D \right) n^i
		   + ...
\nonumber\\[4mm]
&+&
\left( \bxi_{Ri}\,\,  \bxi_{Li} \right)
		\left( \begin{matrix}
			i\,\pt_L  &  
			i\,\sqrt{2}\, \Bigl(\sigma - \frac{\displaystyle m^i}{\displaystyle \sqrt{2}}\Bigr) \\
			i\,\sqrt{2}\,\Bigl(\ov{\sigma} - \frac{\displaystyle \ov{m}{}^i}{\displaystyle\sqrt{2}}\Bigr) &  
			i\,\p_R 
		     \end{matrix} \right)
		\left( \begin{matrix}
			\xi_R^i \\[5mm] \xi_L^i
		     \end{matrix} \right)+ ... ,
\eeqn
where the ellipses denote terms which contain neither $n$ nor $\xi$ fields.
Integrating over the $n$, $\xi$ fields produces the determinant
\beq
	\prod_{i=1}^{N-1}\,\, \,
	\frac{ \det \left( -\pt_k^2 + 
			2\,\Bigl| \sigma - \frac{\displaystyle m^i}{\displaystyle \sqrt{2}} \Bigr|^2 \right) }
             { \det \left( -\p_k^2 + 
			2\,\Bigl| \sigma - \frac{\displaystyle m^i}{\displaystyle \sqrt{2}} \Bigr|^2 + i\, D \right)} \,.
\label{determ}
\eeq
Assuming $\sigma$ and $D$ to be constant background fields,
introducing the notation
\beq
n^N \equiv n \,,\qquad m^N  \equiv  m_0\,,
\label{nnotat}
\eeq
and  evaluating this determinant, 
one arrives at the following effective potential:
\beqn
	V_\text{eff} & =& \int d^2x 
		\left\{  \left( iD ~+~ 2\,\Bigl|\sigma - \frac{m_0}{\sqrt{2}}\Bigr|^2 \right) |n|^2 \right.
	\nonumber\\[4mm]
	&-& 
	\frac{1}{4\pi}\, \sum_{i=1}^{N-1} \Bigl( iD ~+~ 2\,\Bigl|\sigma - \frac{m^i}{\sqrt{2}}\Bigr|^2 \Bigr)\,
		\ln\, \frac{ iD \,+\, 2 | \sigma - \frac{\displaystyle m^i}{\displaystyle \sqrt{2}} |^2} {\Lambda^2}
\nonumber\\[4mm]
	&+&
	\frac{1}{4\pi}\, \sum_{i=1}^{N-1} 2\,\Bigl|\sigma - \frac{m^i}{\sqrt{2}}\Bigr|^2\,
			\ln\, \frac{ 2\, | \sigma - \frac{\displaystyle m^i}{\displaystyle \sqrt{2}} |^2 } { \Lambda^2 }
	+
	\frac{1}{4\pi}\, iD\, (N-1) 
+
	4\, |\omega|^2\,\bigl|\sigma\bigr|^2 \Biggr .
	\nonumber\\
	\label{Veff}
\eeqn
Now, to find the vacua, we must minimize the 
effective potential (\ref{Veff}) with respect to $n$, $D$ and $ \sigma $. In this way we one arrive at
the set of	the vacuum equations,
\beqn
\label{eff1}
	&&
	|n|^2  -  \frac{1}{4\pi} \,\sum_i^{N-1}\, 
		{\rm ln}\, 
		\frac{i D + 2\, \Bigl| \sigma -  \frac{\displaystyle m^i}{\displaystyle \sqrt{2}} \Bigr|^2}
							{\Lambda^2}   = 0\,, 
\\[3mm]				
	&&
	\left( i D ~~+~~ 2\,\Bigl|\sigma - \frac{m_0}{\sqrt{2}}\Bigr|^2 \right) n = 0\,, 
	\label{eff2}
	\\[3mm]
	&&
	\Bigl(\sigma - \frac{m_0}{\sqrt{2}}\Bigr)|n|^2 
		-
	\frac{1}{4\pi} \sum_i^{N-1}\,
			\Bigl(\sigma - \frac{m^i}{\sqrt{2}}\Bigr)\,
		{\rm ln}\, 
		\frac{i D + 2\, \Bigl| \sigma \,-\, \frac{\displaystyle m^i}{\displaystyle \sqrt{2}} \Bigr|^2}
		{2\, \Bigl| \sigma - \frac{\displaystyle m^i}{\displaystyle \sqrt{2}} \Bigr|^2}
		+ 2\, |\omega|^2\sigma ~~=~~ 0\,.
\label{eff3}
\nonumber\\
\eeqn
From Eq.~(\ref{eff2}) it is obvious that there are two options, either
\beq
\label{higgsph}
	 iD + 2\,\left| \sigma - \frac{ m_0}{\sqrt{2}}\right|^2 = 0  
\eeq
or
\beq
\label{strongph}
	 n = 0 \,. 
\eeq
	These two distinct solutions correspond to the Higgs and the strong-coupling regimes of the theory. 
	
As a first step, we solve these equations perturbatively, assuming $ |\omega|^2 $ to be a small parameter,
\beqn
	n &=& \nz  + |\omega|^2\,\no + \ldots\,,
	\nonumber
	 \\[3mm]
	iD &=& i\Dz  + |\omega|^2\,i\Do + \ldots\,, 
	\nonumber
	\\[3mm]
	\sigma &=& \sigz + |\omega|^2\,\sigo + \ldots\,.
\eeqn
	Here $ \nz $, $ \Dz $ and $\sigz$ constitute the solution of the \ntwot CP($N-1$) sigma model,
	in particular $ \Dz = 0 $ in both phases.

	To obtain simple expressions for the solution we will assume the masses to be sitting
	on the circle,
\[
	m^k ~~=~~ m \cdot e^{i 2\pi k / N }\,, \qquad\qquad k ~=~ 0,...\, N-1\,.
\]

%%%%%%%%%%%%%%%%%%%%%%%%%%%%%%%%%%%%%%%%%%%%%%%%%%%%%%%%%%%%%%%%%%%%%%%%%%%%%%%%%%
%
%	  		        subS E C T I O N
%
%%%%%%%%%%%%%%%%%%%%%%%%%%%%%%%%%%%%%%%%%%%%%%%%%%%%%%%%%%%%%%%%%%%%%%%%%%%%%%%%%%
\subsection{The Higgs regime}

	The large-$N$ supersymmetric solution of the \ntwot CP($N-1$) sigma model
	in the Higgs phase is
\begin{align*}
%
	\nz & ~~=~~ \sqrt{\rrenz}\,,   \qquad\qquad\qquad\text{\it the phase of $\nz$ is not determined}\\[3mm]
%
	i\Dz & ~~=~~ ~~~\,0\,, \\[2mm]
%
	\sigz & ~~=~~ ~~\frac{m_0}{\sqrt{2}}\,,
\end{align*}
	where $ \rrenz $ is the renormalized coupling of the unperturbed theory,
\[
	\rrenz ~~=~~ \frac{N}{2\pi}\,{\rm log}\,m/\Lambda\,.
\]

	Expanding equations \eqref{eff1}-\eqref{eff3} to the first order in $|\omega|^2$, we obtain
\begin{align}
%
\notag
	&
	\bnz\,\no ~+~ \text{h.c.} ~~=~~
	\frac{1}{4\pi}\, \sum_{i}^{N-1}\, 
		\frac{ i\Do ~+~ 2\lgr \bsigz\, \sigo ~-~ \bsigo\,\frac{\displaystyle m^i}{\displaystyle \sqrt{2}}
				~+~ \text{h.c.} \rgr }
			{ 2 \Bigl| \sigz - \frac{\displaystyle m^i}{\sqrt{\displaystyle 2}} \Bigr|^2 }\,, \\
%
\label{higgseq}
	&
	i\Do ~~=~~ 0\,, \\[2mm]
%
\notag
	&
	\sigo\,|\nz|^2  ~-~ 
	\frac{1}{4\pi}\,i\Do\, \sum_{i}^{N-1}\, \frac{1}
				{2\Bigl( \bsigz - \frac{\displaystyle \ov{m}{}^i}{\displaystyle \sqrt{2}} \Bigr) }
	~+~ 2\,\sigz ~~=~~ 0\,.
\end{align}
	We can see that $ iD $ vanishes to the first order in $ |\omega|^2 $.
	Thus, for $ iD $ the first order of expansion is not sufficient, since we know that supersymmetry
	is broken and hence $ iD ~\neq~ 0 $.
	This variable, however, is easy to recover in the Higgs phase given the corresponding expansion of $ \sigma $
	via Eq.~\eqref{higgsph}.

	The solution to equations \eqref{higgseq} can be written as
\begin{align*}
%
	i \Dz & ~~=~~ 0\,,                     &       i\Do & ~~=~~ 0\,,  
			\qquad\qquad iD^{(2)} ~~=~~ -\,2\,|\sigo|^2\,, \\[3mm]
%
	\sigz & ~~=~~ \frac{m_0}{\sqrt{2}}\,,  &       \sigo & ~~=~~ -\,\frac{2\sigz}{|\nz|^2}\,, \\[3mm]
%
	|\nz|^2 & ~~=~~ \rrenz\,,              &    
			\no & ~~=~~ -\,\frac{2\,m^0}{\bnz\,|\nz|^2}\,\frac{1}{4\pi}\,
					\sum_{i}^{N-1} \frac{1}{m^0 - m^i}\,.
\end{align*}	

	To simplify these expressions, as noted before, we put the masses on the circle, which gives then
\[
	\sum_{i=1}^{N-1} \frac{1}{m^0 - m^i} ~~=~~ \frac{N-1}{2m} ~~=~~ \frac{N}{2m} ~~+~~ O(1)\,.
\]
	Finally, we obtain 
\begin{align*}
%
	\sigma & ~~=~~ \frac{m^0}{\sqrt{2}}\, \lgr 1 ~~-~~ \frac{2\,|\omega|^2}{|\nz|^2} \rgr ~~+~~ \ldots \,,
	\\[2mm]
%
	i D & ~~=~~ -\, 4\, \frac{m_0^2}{(\rrenz)^2}\, |\omega|^4 ~~+~~ \ldots\,,
	\\[2mm]
%
	n & ~~=~~ \sqrt{\rrenz} ~~-~~ \frac{N}{4\pi}\,\frac{1}{\rrenz\,\bnz}\,|\omega|^2 ~~+~~ \ldots
\end{align*}
	for the Higgs phase, where
\[
	\rrenz ~~=~~ \frac{N}{2\pi}\,{\rm log}\,m/\Lambda\,.
\]

%%%%%%%%%%%%%%%%%%%%%%%%%%%%%%%%%%%%%%%%%%%%%%%%%%%%%%%%%%%%%%%%%%%%%%%%%%%%%%%%%%
%
%	  		        subS E C T I O N
%
%%%%%%%%%%%%%%%%%%%%%%%%%%%%%%%%%%%%%%%%%%%%%%%%%%%%%%%%%%%%%%%%%%%%%%%%%%%%%%%%%%
\subsection{Strong coupling}

	The zeroth order in $ |\omega|^2 $ solution is 
\begin{align*}
%
	\nz & ~~=~~ 0\,, \\[2mm]
%
	i \Dz & ~~=~~ 0\,, \\[2mm]
%
	\sigz & ~~=~~ \wt{\Lambda}\cdot e^{i\frac{2\pi l}{N}}\,,  
	\qquad\qquad\qquad
	\wt{\Lambda} ~~=~~ \sqrt[N]{ \Lambda^N ~+~ m^N }\,,
\end{align*}
	for some fixed $ l = 0,...\,N-1 $.
	Furthermore, in this phase  $ n $ is known exactly,
\[
	n ~~=~~ 0\,.
\]
	The rest two equations in \eqref{eff1}-\eqref{eff3} give at the first order in $ |\omega|^2 $
\begin{align*}
%
	&
	\sum_{i}^{N-1}\, \frac{ i \Do ~+~ 2 \lgr \bsigz\,\sigo ~-~ 
					\sigo\, \frac{\ov{\displaystyle m}{}^i}{\sqrt{\displaystyle 2}}
				~+~ \text{h.c.} \rgr }
			    { 2\, \Bigl| \sigz - \frac{\displaystyle m^i}{\sqrt{\displaystyle 2}} \Bigr|^2 }
			~~=~~ 0\,,
	\\[2mm]
%
	&
	\frac{1}{4\pi}\, \sum_i^{N-1}\, \frac{i \Do}
				    {2\,\Bigl( \bsigz - \frac{\displaystyle\ov{m}{}^i}{\displaystyle \sqrt{2}} \Bigr) }
			~~=~~ 2\, \sigz\,.
\end{align*}

	The solution to these equations are given by
\begin{align}
%
\notag
	n & ~~=~~ 0 \,,  \qquad\qquad\qquad\qquad i \Dz ~~=~~ 0\,,
	\\[3mm]
%
\label{strongeq}
	i \Do & ~~=~~ 8\pi \cdot \frac{ 2\sigz } 
			{\displaystyle \sum_{i}^{N-1} 
				\frac{1}{ \bsigz - \ov{m}{}^i/\sqrt{2}} } \,,
	\qquad\qquad\qquad i ~=~ 1,...\, N-1\,,
	\\[-8mm]
%
\notag
	\sigo & \cdot \sum_{i}^{N-1} \frac{1}
			{ \sigz - \frac{\displaystyle m^i}{\displaystyle\sqrt{2}} }
		~+~ \text{h.c.} ~~=~~
	-\, 8\pi \cdot \sigz \cdot 
	\frac{ \displaystyle \sum_{i}^{N-1} 
			\frac{1}{ \Bigl| \sigz - \frac{\displaystyle m^i}{\displaystyle\sqrt{2}} \Bigr|^2 } }
	  { \displaystyle \sum_{i}^{N-1} 
				\frac{1}{ \bsigz - \frac{\displaystyle \ov{m}{}^i}
							{\displaystyle \sqrt{2}} } }\,.
\end{align}

	We use the following relations to simplify the above solution in the case when the masses 
	are distributed on a circle,
\begin{align*}
%
	&
	\sum_{k=0}^{N-1} \frac{1} { 1 \,-\, \alpha\,e^{\frac{2\pi i k}{N}} } ~~=~~
		\frac{N}{1 ~-~ \alpha^N}\,,
	\\
%
	&
	\sum_{k=0}^{N-1} \frac{1} { (1 + \alpha^2) ~-~ 2\,\alpha \cos \frac{2 \pi k}{N} }
		~~=~~
	\frac{1} {1 \,-\, \alpha^2} 
	\lgr \frac{2\,N}{ 1 \,-\, \alpha^N } ~-~ N \rgr .
\end{align*}

	This gives,
\begin{align*}
%
	& 
	\sum_{k=1}^{N-1} \frac{ 1 }
		{ \sigz - \frac{\displaystyle m^k}{\displaystyle \sqrt{2}} } ~~=~~ 
		 - \, \frac{ 1 }
		{ \sigz - \frac{\displaystyle m}{\displaystyle \sqrt{2}} } ~+~
		N\, \frac{\wt{\Lambda}{}^N}{\Lambda^N}\, \frac{1}{\sigz} \,,
	\\
%
	&
	\sum_{k=1}^{N-1} \frac{ 1 }
                { \Bigl| \sigz - \frac{\displaystyle m^k}{\displaystyle \sqrt{2}} \Bigr|^2 } ~~=~~
	-\, \frac{1}
	{ \Bigl| \sigz - \frac{\displaystyle m}{\displaystyle \sqrt{2}} \Bigr|^2 } 
	~+~
	\frac{2\,N}{ \wt{\Lambda}{}^2 - m^2 } \cdot
	\frac{ \wt{\Lambda}{}^N + m^N } { \Lambda^N }\,.
\end{align*}

	In fact we will only need the leading-N contribution from them above relations.
	In particular, if we substitute these relations into Eq.~\eqref{strongeq} directly,
	we will not have $ iD $ real.
	The reason {\it perhaps} is that the $ \sigz $ solution is only valid up to $ O(1/N) $
	contributions.
	Therefore, we ignore the O(1) contributions versus O(N).
\begin{align*}
%
	i \Do & ~~=~~ \frac{16\pi}{N}\, \frac{\Lambda^N}{\wt{\Lambda}{}^N}\, |\sigz|^2\,,
	\\[3mm]
%
	| \sigo | & ~~=~~
	-\, \frac{8\pi}{N}\, \frac{\Lambda^N}{\wt{\Lambda}{}^N}\,
		\frac{ \wt{\Lambda}{}^3 } {\wt{\Lambda}{}^2 - m^2 }\,
		\Bigl[ 1 ~+~ \frac{m^N}{\wt{\Lambda}{}^N} \Bigr] \cdot
		\Bigl( \cos \bigl( \frac{2\pi l }{N} - \varphi \bigr) \Bigr)^{-1}\,,
\end{align*}
	where $ \varphi $ is the arbitrary phase of $ \sigo $.
	We can choose this phase to be the same as that of $ \sigz $, 
\[
	\varphi ~~\equiv~~ \frac{2\pi l}{N}
\]
	for the cosine to disappear.
	If we take the limit of small masses
\[
	\frac{m^N}{\Lambda^N} ~~\ll~~ 1\,,
\]
	and also note that
\[
	\wt{\Lambda} ~~=~~ \Lambda 
\]
	with exponential accuracy in $ 1/N $,
	we can further simplify the result, and arrive at
\begin{align*}
%
	n & ~~=~~ 0 
	\\[3mm]
%
	i D &  ~~=~~ |\omega|^2 \cdot \frac{16\pi}{N}\cdot \Lambda^2 ~+~ \ldots \,, 
	\qquad\qquad\qquad \text{for }\frac{m^N}{\Lambda^N} ~\ll~ 1\,,
	\\[3mm]
%
	\sigma & ~~=~~ \Lambda \cdot e^{\frac{2\pi i l}{N}} 
			~-~ |\omega|^2\, \frac{8\pi}{N}\, 
				\frac{\Lambda^3}{\Lambda^2 - m^2}\, e^{\frac{2\pi i l}{N}}
	\,.
\end{align*}

	Note that although $ |\omega|^2 $ grows as $ O(N) $ for large $N$, the coefficients of 
	$|\omega|^2$-corrections are suppressed by the corresponding power of $ 1/N $ so that 
	the corrections are neutral in $ N $.




\newpage

\section{Remarks on mirror for heterotic CP(1) }



The geometric representatition of  the heterotic CP(1) model is the following deformation
of the ${\mathcal N}=(2,2)$ model \cite{SY1}:
\beqn
&&
L_{{\rm heterotic}}= 
\zeta_R^\dagger \, i\partial_L \, \zeta_R  + 
\left[\gamma \, \zeta_R  \,R\,  \big( i\,\partial_{L}\phi^{\dagger} \big)\psi_R
+{\rm H.c.}\right] -g_0^2|\gamma |^2 \left(\zeta_R^\dagger\, \zeta_R
\right)\left(R\,  \psi_L^\dagger\psi_L\right)
\nonumber
\\[4mm]
&&
+
G\, \left\{\rule{0mm}{5mm}
\partial_\mu \phi^\dagger\, \partial^\mu\phi  
+\frac{i}{2}\big(\psi_L^\dagger\!\stackrel{\leftrightarrow}{\partial_R}\!\psi_L 
+ \psi_R^\dagger\!\stackrel{\leftrightarrow}{\partial_L}\!\psi_R
\big)\right.
\nonumber
\\[4mm] 
&&
-\frac{i}{\chi}\,  \big[\psi_L^\dagger \psi_L
\big(\phi^\dagger \!\stackrel{\leftrightarrow}{\partial_R}\!\phi
\big)+ \psi_R^\dagger\, \psi_R
\big(\phi^\dagger\!\stackrel{\leftrightarrow}{\partial_L}\!\phi
\big)
\big]
-
\frac{2(1- g_0^2 |\gamma |^2)}{\chi^2}\,\psi_L^\dagger\,\psi_L \,\psi_R^\dagger\,\psi_R
\Big\}\,,
\label{AAone}
\eeqn
where the field $\zeta_R$ appearing in the first line is
the spinor field on $C$ which appears, with necessity in the ${\mathcal N}=(0,2)$ model \cite{EdTo}.
Here $G$ is the metric while $R$ is the Ricci tensor.
To begin with, we let us assume the deformation parameter $\gamma$ to be small (it is dimensionless)
and work to the leading order in $\gamma$.

Note that 
\beqn
&&
K\big|_{\theta=\bar\theta=0}=\frac{2}{g_{0}^{2}}\,\ln \chi\,, 
\nonumber
\\[2mm]
&&
G=G_{1\bar 1}=\partial_\phi\partial_{\phi^\dagger\,} 
K\big|_{\theta=\bar\theta=0}=
\frac{2}{g_{0}^2\,\chi^{2}}\,,
\nonumber
\\[2mm]
&&
\Gamma =\Gamma^{1}_{11} =- 2\, \frac{\phi^\dagger\,}{\chi}\,,\quad 
\bar\Gamma =\Gamma^{\bar 1}_{\bar 1\bar 1}=
- 2\, \frac{ \phi}{\chi}\,,
\nonumber
\\[2mm]
&&
 R \equiv  R_{1\bar 1}=-G^{-1}\!R_{1\bar 1 1\bar 1}=\frac{2}{\chi^2}\,,
\label{Atwo}
\eeqn
where the following notation
is implied:
\beq
\chi \equiv 1+\phi\,\phi^\dagger\,.
\eeq
Thus, the kinetic terms of the CP(1) fields $\phi$ and $\psi$ contain $\frac{1}{g^2}$ in the normalization
while $\gamma$ in the first line is defined in conjunction with the Ricci tensor, so that there is no $\frac{1}{g^2}$
in front of this term. This convention is important for what follows.

Now, let us remember that the undeformed CP(1) model has a mirror representation \cite{MR1,MR2}
\beq
{\mathcal L}_{\rm mirror} = \int d^4\theta\, Y^\dagger\, Y + \int d^2\theta\, {\mathcal W}_{\rm mirror}
(Y) + {\rm H.c.}
\label{five}
\eeq
where 
\beq
{\mathcal W}_{\rm mirror} = \Lambda \left( e^Y + e^{-Y}\right)\,,
\eeq
and $\Lambda$ is the dynamical scale of the CP(1) model.

The question is: ``what is the mirror representation of the deformed model (\ref{AAone}), to the leading order in $\gamma$?"

Surprisingly, this question has a very simple answer.
To find the answer let us observe that the term of the first order in $\gamma$ in (\ref{AAone})
is nothing but the superconformal anomaly in the unperturbed CP(1) model (it is sufficient to consider this anomaly in the unperturbed model since we are after the leading term in $\gamma$ in the mirror representation).
More exactly, in the CP(1) model
\beq
\gamma_\mu J^\mu_{\,\,\alpha} = -\frac{ \sqrt 2}{2\pi}\,R\, \left(\partial_\nu \phi^\dagger\right)\left(
\gamma^\nu\psi\right)_\alpha\,,
\label{six}
\eeq
where $J^\mu_{\,\,\alpha}$ is the supercurrent.
In what follows, for simplicity,  numerical factors like $2$ or $\pi$ will be omitted
This means that the deformation term in (\ref{AAone}) can be written as
\beq
\Delta {\mathcal L} = \gamma \zeta_R\left(\gamma_\mu \, J^\mu\right)_L
\label{seven}
\eeq 

Since (\ref{six}) has a geometric meaning we can readily rewrite this term in the mirror representation
(\ref{five}). Indeed, in the generalized Wess-Zumino model
\beq
\left(\gamma_\mu \, J^\mu\right)_L
 = \frac{\partial {\mathcal W}_{\rm mirror}}{\partial Y}\,\psi_L\,.
 \label{eight}
\eeq
This equation is not quite exact, but it is good enough for the time being.

Now, combining (\ref{seven}) and (\ref{eight})
we get the following deformation of ${\mathcal L}_{\rm mirror}$ (in the leading order in $\gamma$):
\beq
\Delta {\mathcal L}_{\rm mirror}= \gamma \, \zeta_R \,\psi_L\,\, \frac{\partial {\mathcal W}_{\rm mirror}}{\partial Y}
+{\rm H.c.}
\eeq

In superfields this is equivalent to adding to (\ref{five}) the following terms:
\beq
\Delta {\mathcal L}_{\rm mirror}  \int d^4\theta {\mathcal B}\, {\mathcal B}^\dagger 
+\gamma \int d\theta_L^\dagger\, \theta_L \,d\theta_L  d\theta_R\,\, {\mathcal B} \,{\mathcal W}_{\rm mirror}
+{\rm H.c.}
\eeq

This adds up to the scalar potential the term 
\beq
\Delta V = |\gamma|^2 \left|{\mathcal W}_{\rm mirror}\right|^2\,.
\eeq
Note that a constant shift in ${\mathcal W}_{\rm mirror}$
becomes an observable physical effect.

\newpage

\section{Different effective Lagrangians}

\newpage

\section*{Appendix A: \\
Minkowski vesrus Euclidean formulation}
\addcontentsline{toc}{section}{Appendix A: \\
Minkowski vesrus Euclidean formulation}
 \renewcommand{\theequation}{A.\arabic{equation}}
\setcounter{equation}{0}
 
 \renewcommand{\thesubsection}{A.\arabic{subsection}}
\setcounter{subsection}{0}

In the bulk of the paper we use both, Minkowski and Euclidean conventions.
It is useful to summarize the transition rules.
If the Minkowski coordinates are
\beq
x^\mu_M =\{t,\,z\}\,,
\eeq
the passage to the Euclidean space requires
\beq
t \to - i\tau\,,
\eeq
and the Euclidean coordinates are
\beq
x^\mu_M =\{\tau,\,z\}\,.
\eeq
The derivatives are defined as follows:
\beqn
\pt_L^M &=& \pt_t+\pt_z\,,\qquad \pt_R^M = \pt_t- \pt_z\,,
\nonumber\\[2mm]
\pt_L^E &=& \pt_\tau - i \pt_z\,,\qquad \pt_R^M = \pt_\tau + i \pt_z\,.
\eeqn
The Dirac spinor is
\beq
\Psi =\left(
\begin{array}{c}
\psi_R\\[1mm]
\psi_L
\end{array}
\right)
\eeq
In passing to the Eucildean space $\Psi^M = \Psi^E$;
however, $\bar\Psi$ is transformed,
\beq
\bar\Psi^M \to i \bar \Psi^E\,.
\eeq
Moreover, $\Psi^E$ and $\bar \Psi^E$ are {\em not} related by the complex conjugation operation.
They become independent variables. The fermion gamma matrices are defined as
\beq
\bar\sigma^\mu_M =\{1,\,-\sigma_3\}\,,\qquad \bar\sigma^\mu_E =\{1,\, i\sigma_3\}\,.
\eeq
Finally, 
\beq
\cell_E =- \cell_M (t=-i\tau , ...).
\eeq
With this notation, formally, the fermion kinetic terms in $\cell_E $ and $\cell_M $
coincide.
If we want the heterotic deformation term to have the same form in
$\cell_E $ and $\cell_M $ we must (and do) transform the heterotic deformation parameter as follows:
\beq
\gamma_M = i\,\gamma_E\,.
\eeq

Everywhere where there is no menace of confusion we omit the super/sub\-scripts $M,E$.
It is obvious from the context where the Euclidean or Minkowski formulation is implied.


\newpage

\section*{Appendix B: \\
Global symmetries of the CP\boldmath{$(N-1)$} model with \boldmath{$Z_{2N}$}-symmetric
 twisted masses}
\addcontentsline{toc}{section}{Appendix: \\
Global symmetries of the CP$(N-1)$ model with  $Z_N$-symmetric
 twisted masses}
 \renewcommand{\theequation}{B.\arabic{equation}}
\setcounter{equation}{0}
 
 \renewcommand{\thesubsection}{B.\arabic{subsection}}
\setcounter{subsection}{0}

%First, let us outline the \ntwo CP$(N-1)$ model with twisted masses
%\cite{Alvarez} in one
%of a few possible formulations, the so-called gauge formulation \cite{peregolo}.
%This formulation is built on an $N$-plet of complex scalar fields $n^i$ where $i=1,2,...,N$.
%We impose the constraint
%\beq
%n_i^\dagger\,n^i = 1\,.
%\eeq
%This leaves us with $2N-1$ real bosonic degrees of freedom. To eliminate one extra degree
%of freedom we impose a local U(1) invariance $n^i(x)\to e^{i\alpha(x)} n^i(x)$.
%To this end we introduce a gauge field $A_\mu$ which converts the partial derivative into the
%covariant one,
%\beq
%\partial_\mu\to \nabla_\mu \equiv \partial_\mu -i\,  A_\mu\,.
%\eeq
%The field $A_\mu$ is auxiliary; it enters in the Lagrangian without derivatives. The kinetic term of the
%$n$ fields is
%\beq
%L =\frac{2}{g_0^2}\, \left|\nabla_\mu n^i\right|^2\,.
%\eeq

%The superpartner to the field $n^i$ is an $N$-plet of complex two-component spinor fields $\xi^i$,
%\beq
%\xi^i =\left\{\begin{array}{l}
%\xi^i_R\\[2mm]
%\xi^i_L
%\end{array}
%\right.\,.
%\eeq

%The auxiliary field $A_\mu$ has a complex scalar superpartner $\sigma$ 
%and a two-component complex spinor superpartner $\lambda$; both enter without derivatives.
%The full \ntwo symmetric Lagrangian is
%\beqn
%L &=& \frac{2}{g_0^2}\left\{\rule{0mm}{6mm}
%\left|\nabla_\mu n^i\right|^2+ \xi_i^\dagger\, i\gamma^\mu\nabla_\mu\,\xi^i
%+ 2\sum_i\left|\sigma-\frac{m_i}{\sqrt 2}\right|^2\, |n^i|^2\right.
%\nonumber\\[3mm]
%&+&\left.\left[
%i\sqrt{2}\,\sum_i \left( \sigma -\frac{m_i}{\sqrt 2}\right)\xi^\dagger_{iR}\, \xi^i_L +i\sqrt{2}\,n^\dagger_i
%\left(\lambda_R\xi^i_L - \lambda_L\xi^i_R
%\right)+{\rm H.c.}\right]
%\right\}.
%\nonumber\\
%\label{bee31}
%\eeqn
%where $m_i$ are twisted mass parameters.
%Equation (\ref{bee31}) is valid in a special case when
%\beq
%\sum_{i=1}^N \,m_i = 0\,.
%\label{bee32}
%\eeq
%We will make a specific choice of the
%parameters $m_i$, namely,
%\beq
%m_i= m  \left\{  e^{2\pi i/N} , \,e^{4\pi i/N},\, ... ,
% \,e^{2(N-1)\pi i/N} , \, 1\right\} \,,
%\label{spch}
%\eeq
%where $m$ is a single common parameter. Then the constraint
%(\ref{bee32}) is automatically satisfied. Without loss of generality $m$ can be assumed
%to be real and positive. The U(1) gauge symmetry is built in. This symmetry eliminates one bosonic degree of freedom, leaving us with $2N-2$ dynamical bosonic degrees of freedom inherent to CP$(N-1)$ model.

Now let us discuss global symmetries of this model. In the absence of the twisted masses
the model was SU$(N)$ symmetric. The twisted masses (\ref{spch}) explicitly break this symmetry down to U$(1)^{N-1}$,
\beqn
n^\ell&\to& e^{i\alpha_\ell}n^\ell\,,\quad \xi^\ell_R \to e^{i\alpha_\ell}\xi^\ell_R\,
\quad \xi^\ell_L \to e^{i\alpha_\ell}\xi^\ell_L\,,\quad \ell=1,2, ..., N\,,
\nonumber\\[2mm]
\sigma
&\to&
 \sigma\,,\quad \lambda_{R,L}\to \lambda_{R,L}\,.
\eeqn
where $\alpha_\ell$ are $N$ constant phases different for different $\ell$. 

Next, there is a global vectorial U(1) symmetry which rotates all fermions $\xi^\ell$
in one and the same way, leaving the boson fields intact,
\beqn
\xi^\ell_R 
&\to& 
e^{i\beta}\xi^\ell_R\,, \quad
 \xi^\ell_L \to e^{i\beta}\xi^\ell_L\,,\quad \ell=1,2, ..., N\,,
\nonumber\\[2mm]
\lambda_R 
&\to&
 e^{-i\beta}\lambda_R\,,\quad 
\lambda_L \to e^{-i\beta}\lambda_L\,,
\nonumber\\[2mm]
n^\ell &\to& n^\ell\,,\quad \sigma\to\sigma\,.
\eeqn

Finally, there is a discrete $Z_{2N}$ symmetry which is of most importance for our purposes.
Indeed, let us start from the axial U$(1)_R$ transformation which would be a symmetry
of the classical action at $m=0$ 
 (it is anomalous, though, under quantum corrections),
\beqn
\xi^\ell_R 
&\to& 
e^{i\gamma}\xi^\ell_R\,, \quad
 \xi^\ell_L \to e^{-i\gamma }\xi^\ell_L\,,\quad \ell=1,2, ..., N\,,
 \nonumber\\[2mm]
 \lambda_R 
&\to&
 e^{i\gamma}\lambda_R\,,\quad 
 \lambda_L \to e^{-i\gamma}\lambda_L\,,\quad \sigma \to e^{2i\gamma}\sigma\,,
\nonumber\\[2mm]
n^\ell
&\to&
 n^\ell\,.
\eeqn
With $m$ switched on and the chiral anomaly included, this transformation 
is no longer the symmetry of the model. However, a discrete $Z_{2N}$ subgroup survives both the inclusion of anomaly and $m\neq 0$. This subgroup corresponds to
\beq
\gamma_k =\frac{2\pi i k}{2N}\,,\quad k= 1,2, ..., N\,.
\eeq
with the simultaneous shift
\beq
\ell\to \ell - k\,.
\eeq
In other words,
\beqn
\xi^\ell_R 
&\to& 
e^{i\gamma_k}\xi^{\ell-k}_R\,, \quad
 \xi^\ell_L \to e^{-i\gamma_k }\xi^{\ell-k}_L\,, 
 \nonumber\\[2mm]
 \lambda_R 
&\to&
 e^{i\gamma_k}\lambda_R\,,\quad 
 \lambda_L \to e^{-i\gamma_k}\lambda_L\,,\quad \sigma \to e^{2i\gamma_k}\sigma\,,
 \nonumber\\[2mm]
 n^\ell &\to & n^{\ell-k}\,.
 \label{bee35}
\eeqn
This $Z_{2N}$ symmetry  relies on the particular choice of masses 
given in (\ref{spch}).

The order parameters for the $Z_N$ symmetry are as follows:
(i) the set of the vacuum expectation values
$\{ \langle n^1\rangle,\,\, \langle n^2\rangle, \,...\, \langle n^N\rangle\}$
and (i) the bifermion condensate $\langle  \xi^\dagger_{R,\,\ell}\xi^\ell_L\rangle$.
Say, a nonvanishing value of $\langle n^1\rangle$ or  $\langle  \xi^\dagger_{R,\,\ell}\xi^\ell_L\rangle$ implies that the $Z_{2N}$ symmetry of the action is broken down to
$Z_2$. The first order parameter is more convenient for detection
at large $m$ while the second at small $m$. 

It is instructive to illustrate the above conclusions
in a different formulation of the sigma model, namely, in the geometrical formulation
(for simplicity we will consider CP(1); generalization to CP$(N-1)$ is straightforward).
In components the Lagrangian of the model is
\beqn
&&
{\mathcal L}_{\,  CP(1)}= G\, \Big\{
\partial_\mu \bar{\phi}\, \partial^\mu\phi -|m|^2{\bar{\phi}\,\phi} 
+\frac{i}{2}\big(\psi_L^\dagger\!\stackrel{\leftrightarrow}{\partial_R}\!\psi_L 
+ \psi_R^\dagger\!\stackrel{\leftrightarrow}{\partial_L}\!\psi_R
\big)
\nonumber\\[1mm] 
&&
-i\,\frac{1-\bar{\phi}\,\phi}{\chi} \,\big(m\,\psi_L^\dagger \psi_R + \bar m
\psi_R^\dagger \psi_L
\big)
\nonumber\\[1mm] 
&&
-\frac{i}{\chi}\,  \big[\psi_L^\dagger \psi_L
\big(\bar{\phi} \!\stackrel{\leftrightarrow}{\partial_R}\!\phi
\big)+ \psi_R^\dagger\, \psi_R
\big(\bar{\phi}\!\stackrel{\leftrightarrow}{\partial_L}\!\phi
\big)
\big]
\nonumber\\[1mm]
&&
-
\frac{2}{\chi^2}\,\psi_L^\dagger\,\psi_L \,\psi_R^\dagger\,\psi_R
\Big\}\,,
\label{Aone}
\eeqn
where 
\beq
\chi = 1+\bar{\phi}\,\phi\,,\quad G= \frac{2}{g_0^2\,\chi^2}\,.
\eeq
The $Z_2$ transformation corresponding to (\ref{bee35}) is
\beq
\phi \to -\frac{1}{\bar{\phi}}\,,\qquad \psi_R^\dagger \psi_L\to -
\psi_R^\dagger \psi_L\,.
\label{bee40}
\eeq
 The order parameter which can detect breaking/nonbreaking of the above
symmetry is
\beq
\frac{m}{g_0^2} \left(1- \frac{g_0^2}{2\pi}
\right)\, \frac{\bar{\phi}\,\phi-1}{\bar{\phi}\,\phi+1} - 
i R \psi_R^\dagger \psi_L\,.
\eeq
Under the transformation (\ref{bee40}) this order parameter changes sign.
In fact, this is  the central charge of the \ntwo
sigma model, including the anomaly  \cite{svz}.

Now, what changes if instead of the \ntwo model we will consider nonsupersymmetric 
CP$(N-1)$ model with twisted masses? Then the part of the Lagrangian (\ref{bee31}) containing fermions must be dropped. The same must be done in the $Z_2$ order parameter.
As was shown in~\cite{GSY05,GSYphtr}, now at $m>\Lambda$ the $Z_2$ symmetry is broken, while at $m<\Lambda$ unbroken. A phase transition takes place.





\newpage
\vspace{2.5cm}

\small
\begin{thebibliography}{99}
\itemsep -2pt

\bibitem{ABEKY}
R.~Auzzi, S.~Bolognesi, J.~Evslin, K.~Konishi and A.~Yung,
%{\em Non-Abelian superconductors: Vortices and
%confinement in N = 2
%SQCD,}
Nucl.\ Phys.\ B {\bf 673}, 187 (2003)
[hep-th/0307287].
%%CITATION = HEP-TH 0307287;%%

\bibitem{SYmon}
M.~Shifman and A.~Yung,
%``Non-Abelian string junctions as confined monopoles,''
Phys.\ Rev.\ D {\bf 70}, 045004 (2004)
[hep-th/0403149].
%%CITATION = HEP-TH 0403149;%%

\bibitem{HT2}
A.~Hanany and D.~Tong,
%``Vortex strings and four-dimensional gauge dynamics,''
JHEP {\bf 0404}, 066 (2004)
[hep-th/0403158].
%%CITATION = HEP-TH 0403158;%%

\bibitem{Trev}
D.~Tong,
  %``Quantum Vortex Strings: A Review,''
  Annals Phys.\  {\bf 324}, 30 (2009)
  [arXiv:0809.5060 [hep-th]];
  %%CITATION = APNYA,324,30;%%
  M.~Eto, Y.~Isozumi, M.~Nitta, K.~Ohashi and N.~Sakai,
  %``Solitons in the Higgs phase: The moduli matrix approach,''
  J.\ Phys.\ A  {\bf 39}, R315 (2006)
  [arXiv:hep-th/0602170];
  %%CITATION = JPAGB,A39,R315;%%
  K.~Konishi,
  %``The magnetic monopoles seventy-five years later,''
  Lect.\ Notes Phys.\  {\bf 737}, 471 (2008)
  [arXiv:hep-th/0702102];
  %%CITATION = LNPHA,737,471;%%
M.~Shifman and A.~Yung,
{\sl Supersymmetric Solitons,}
(Cambridge University Press, 2009).
  %%CITATION = HEP-TH/0703267;%%

\bibitem{W79}
E.~Witten,
%``Instantons, The Quark Model, And The 1/N Expansion,''
Nucl.\ Phys.\ B {\bf 149}, 285 (1979).
%%CITATION = NUPHA,B149,285;%%

\bibitem{W93}
E.~Witten,
  %``Phases of N = 2 theories in two dimensions,''
  Nucl.\ Phys.\ B {\bf 403}, 159 (1993)
  [hep-th/9301042].
  %%CITATION = HEP-TH 9301042;%%
  
  \bibitem{dadvl}
  A.~D'Adda, P.~Di Vecchia and M.~Luscher,
  %``Confinement And Chiral Symmetry Breaking In Cp**N-1 Models With Quarks,''
  Nucl.\ Phys.\  B {\bf 152}, 125 (1979).
  %%CITATION = NUPHA,B152,125;%%
  
  \bibitem{SYhet}
  M.~Shifman and A.~Yung,
  %``Large-N Solution of the Heterotic N=(0,2) Two-Dimensional CP(N-1) Model,''
  Phys.\ Rev.\  D {\bf 77}, 125017 (2008)
  [arXiv:0803.0698 [hep-th]].
  %%CITATION = PHRVA,D77,125017;%%
  
  \bibitem{MR1}
  K.~Hori and C.~Vafa,
{\em Mirror symmetry,}
  arXiv:hep-th/0002222.
  %%CITATION = HEP-TH/0002222;%%
  
\bibitem{MR2}
E.~Frenkel and A.~Losev,
  %``Mirror symmetry in two steps: A-I-B,''
  Commun.\ Math.\ Phys.\  {\bf 269}, 39 (2007)
  [arXiv:hep-th/0505131].
  %%CITATION = CMPHA,269,39;%%
  
\bibitem{EdTo}
 M.~Edalati and D.~Tong,
  %``Heterotic vortex strings,''
  JHEP {\bf 0705}, 005 (2007)
  [arXiv:hep-th/0703045].
  %%CITATION = JHEPA,0705,005;%%
  
\bibitem{SY1}
  M.~Shifman and A.~Yung,
  %``Heterotic Flux Tubes in N=2 SQCD with N=1 Preserving Deformations,''
  Phys.\ Rev.\  D {\bf 77}, 125016 (2008)
  [arXiv:0803.0158 [hep-th]].
  %%CITATION = PHRVA,D77,125016;%%
  
  \bibitem{BSY1}
P.~A.~Bolokhov, M.~Shifman and A.~Yung,
  %``Description of the Heterotic String Solutions in U(N) SQCD,''
  Phys. \ Rev. \ D {\bf 79}, 085015 (2009) (Erratum: Phys. Rev. D 80, 049902 (2009))
  [arXiv:0901.4603 [hep-th]].
  %%CITATION = ARXIV:0901.4603;%%
  
    \bibitem{BSY2}
  P.~A.~Bolokhov, M.~Shifman and A.~Yung,
  %``Description of the Heterotic String Solutions in the M Model,''
  Phys. \ Rev. \ D {\bf 79}, 106001 (2009) (Erratum: Phys. Rev. D 80, 049903 (2009))
  [arXiv:0903.1089 [hep-th]].
  %%CITATION = ARXIV:0903.1089;%%  
  
      \bibitem{BSY3}
  P.~A.~Bolokhov, M.~Shifman and A.~Yung,
{\em Heterotic \nzt \mbox{\cpn} Model with Twisted Masses,}
  arXiv:0907.2715 [hep-th].
  %%CITATION = ARXIV:0907.2715;%%
  
      \bibitem{orco}
  E.~Witten,
  %``A Supersymmetric Form Of The Nonlinear Sigma Model In Two-Dimensions,''
  Phys.\ Rev.\  D {\bf 16}, 2991 (1977);
  %%CITATION = PHRVA,D16,2991;%%
  P.~Di Vecchia and S.~Ferrara,
  %``Classical Solutions In Two-Dimensional Supersymmetric Field Theories,''
  Nucl.\ Phys.\  B {\bf 130}, 93 (1977).
  %%CITATION = NUPHA,B130,93;%%

 \bibitem{Bruno}
 B.~Zumino,
  %``Supersymmetry And Kahler Manifolds,''
  Phys.\ Lett.\  B {\bf 87}, 203 (1979).
  %%CITATION = PHLTA,B87,203;%%

 \bibitem{rev1}
 V.~A.~Novikov, M.~A.~Shifman, A.~I.~Vainshtein and V.~I.~Zakharov,
  %``Two-Dimensional Sigma Models: Modeling Nonperturbative Effects Of Quantum
  %Chromodynamics,''
  Phys.\ Rept.\  {\bf 116}, 103 (1984).
  %%CITATION = PRPLC,116,103;%%
   
  \bibitem{rev2} 
 A.~M.~Perelomov,
  %``SUPERSYMMETRIC CHIRAL MODELS: GEOMETRICAL ASPECTS,''
  Phys.\ Rept.\  {\bf 174}, 229 (1989).
  %%CITATION = PRPLC,174,229;%%
  
  \bibitem{WI}
  E.~Witten,
  %``Constraints On Supersymmetry Breaking,''
  Nucl.\ Phys.\  B {\bf 202}, 253 (1982).
  %%CITATION = NUPHA,B202,253;%%
  
  \bibitem{twisted}
 L.~Alvarez-Gaum\'{e} and D.~Z.~Freedman,
%``Potentials For The Supersymmetric Nonlinear Sigma Model,''
Commun.\ Math.\ Phys.\  {\bf 91}, 87 (1983);
%%CITATION = CMPHA,91,87;%%
S.~J.~Gates,
%``Superspace Formulation Of New Nonlinear Sigma Models,''
Nucl.\ Phys.\ B {\bf 238}, 349 (1984);
%%CITATION = NUPHA,B238,349;%%
S.~J.~Gates, C.~M.~Hull and M.~Ro\v{c}ek,
%``Twisted Multiplets And New Supersymmetric Nonlinear Sigma Models,''
Nucl.\ Phys.\ B {\bf 248}, 157 (1984).
%%CITATION = NUPHA,B248,157;%%

\bibitem{BelPo}
 A.~M.~Polyakov,
  %``Interaction Of Goldstone Particles In Two-Dimensions. Applications To
  %Ferromagnets And Massive Yang-Mills Fields,''
  Phys.\ Lett.\  B {\bf 59}, 79 (1975).
  %%CITATION = PHLTA,B59,79;%%
  
  \bibitem{adam}
  A.~Ritz, M.~Shifman and A.~Vainshtein,
  %``Counting domain walls in N = 1 super Yang-Mills,''
  Phys.\ Rev.\  D {\bf 66}, 065015 (2002)
  [arXiv:hep-th/0205083].
  %%CITATION = PHRVA,D66,065015;%%
  
  \bibitem{Dor}
N.~Dorey,
%``The BPS spectra of two-dimensional
%supersymmetric gauge theories
%with  twisted mass terms,''
JHEP {\bf 9811}, 005 (1998) [hep-th/9806056].
%%CITATION = HEP-TH 9806056;%%
  
  \bibitem{WessBagger}
J. Wess and J. Bagger, {\em Supersymmetry and Supergravity}, Second Edition,
Princeton University Press, 1992.

\bibitem{Helgason}
S. Helgason, {\em Differential geometry, Lie groups and symmetric spaces},
Academic Press, New York, 1978.


\end{thebibliography}


\end{document}
