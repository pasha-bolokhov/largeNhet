%\documentclass{article}
\documentclass[12pt]{article}
\usepackage{latexsym}
\usepackage{amsmath}
\usepackage{amssymb}
\usepackage{relsize}
\usepackage{geometry}
\geometry{letterpaper}

\usepackage{showlabels}

\textwidth = 6.0 in
\textheight = 8.5 in
\oddsidemargin = 0.0 in
\evensidemargin = 0.0 in
\topmargin = 0.2 in
\headheight = 0.0 in
\headsep = 0.0 in
%\parskip = 0.05in
\parindent = 0.35in


%% common definitions
\def\stackunder#1#2{\mathrel{\mathop{#2}\limits_{#1}}}
\def\beqn{\begin{eqnarray}}
\def\eeqn{\end{eqnarray}}
\def\nn{\nonumber}
\def\baselinestretch{1.1}
\def\beq{\begin{equation}}
\def\eeq{\end{equation}}
\def\ba{\beq\new\begin{array}{c}}
\def\ea{\end{array}\eeq}
\def\be{\ba}
\def\ee{\ea}
\def\stackreb#1#2{\mathrel{\mathop{#2}\limits_{#1}}}
\def\Tr{{\rm Tr}}
\newcommand{\gsim}{\lower.7ex\hbox{$
\;\stackrel{\textstyle>}{\sim}\;$}}
\newcommand{\lsim}{\lower.7ex\hbox{$
\;\stackrel{\textstyle<}{\sim}\;$}}
\newcommand{\nfour}{${\mathcal N}=4$ }
\newcommand{\ntwo}{${\mathcal N}=2$ }
\newcommand{\ntwon}{${\mathcal N}=2$}
\newcommand{\ntwot}{${\mathcal N}= \left(2,2\right) $ }
\newcommand{\ntwoo}{${\mathcal N}= \left(0,2\right) $ }
\newcommand{\none}{${\mathcal N}=1$ }
\newcommand{\nonen}{${\mathcal N}=1$}
\newcommand{\vp}{\varphi}
\newcommand{\pt}{\partial}
\newcommand{\ve}{\varepsilon}
\newcommand{\gs}{g^{2}}
\newcommand{\qt}{\tilde q}
\renewcommand{\theequation}{\thesection.\arabic{equation}}

%%
\newcommand{\p}{\partial}
\newcommand{\wt}{\widetilde}
\newcommand{\ov}{\overline}
\newcommand{\mc}[1]{\mathcal{#1}}
\newcommand{\md}{\mathcal{D}}

\newcommand{\GeV}{{\rm GeV}}
\newcommand{\eV}{{\rm eV}}
\newcommand{\Heff}{{\mathcal{H}_{\rm eff}}}
\newcommand{\Leff}{{\mathcal{L}_{\rm eff}}}
\newcommand{\el}{{\rm EM}}
\newcommand{\uflavor}{\mathbf{1}_{\rm flavor}}
\newcommand{\lgr}{\left\lgroup}
\newcommand{\rgr}{\right\rgroup}

\newcommand{\Mpl}{M_{\rm Pl}}
\newcommand{\suc}{{{\rm SU}_{\rm C}(3)}}
\newcommand{\sul}{{{\rm SU}_{\rm L}(2)}}
\newcommand{\sutw}{{\rm SU}(2)}
\newcommand{\suth}{{\rm SU}(3)}
\newcommand{\ue}{{\rm U}(1)}
%%%%%%%%%%%%%%%%%%%%%%%%%%%%%%%%%%%%%%%
%  Slash character...
\def\slashed#1{\setbox0=\hbox{$#1$}             % set a box for #1
   \dimen0=\wd0                                 % and get its size
   \setbox1=\hbox{/} \dimen1=\wd1               % get size of /
   \ifdim\dimen0>\dimen1                        % #1 is bigger
      \rlap{\hbox to \dimen0{\hfil/\hfil}}      % so center / in box
      #1                                        % and print #1
   \else                                        % / is bigger
      \rlap{\hbox to \dimen1{\hfil$#1$\hfil}}   % so center #1
      /                                         % and print /
   \fi}                                        %

%%EXAMPLE:  $\slashed{E}$ or $\slashed{E}_{t}$

%%

\newcommand{\LN}{\Lambda_\text{SU($N$)}}
\newcommand{\sunu}{{\rm SU($N$) $\times$ U(1) }}
\newcommand{\sunun}{{\rm SU($N$) $\times$ U(1)}}
\def\cfl {$\text{SU($N$)}_{\rm C+F}$ }
\def\cfln {$\text{SU($N$)}_{\rm C+F}$}
\newcommand{\mUp}{m_{\rm U(1)}^{+}}
\newcommand{\mUm}{m_{\rm U(1)}^{-}}
\newcommand{\mNp}{m_\text{SU($N$)}^{+}}
\newcommand{\mNm}{m_\text{SU($N$)}^{-}}
\newcommand{\AU}{\mc{A}^{\rm U(1)}}
\newcommand{\AN}{\mc{A}^\text{SU($N$)}}
\newcommand{\aU}{a^{\rm U(1)}}
\newcommand{\aN}{a^\text{SU($N$)}}
\newcommand{\baU}{\ov{a}{}^{\rm U(1)}}
\newcommand{\baN}{\ov{a}{}^\text{SU($N$)}}
\newcommand{\lU}{\lambda^{\rm U(1)}}
\newcommand{\lN}{\lambda^\text{SU($N$)}}
%\newcommand{\Tr}{{\rm Tr\,}}
\newcommand{\bxir}{\ov{\xi}{}_R}
\newcommand{\bxil}{\ov{\xi}{}_L}
\newcommand{\xir}{\xi_R}
\newcommand{\xil}{\xi_L}
\newcommand{\bzl}{\ov{\zeta}{}_L}
\newcommand{\bzr}{\ov{\zeta}{}_R}
\newcommand{\zr}{\zeta_R}
\newcommand{\zl}{\zeta_L}
\newcommand{\nbar}{\ov{n}}

\newcommand{\CPC}{CP($N-1$)$\times$C }
\newcommand{\CPCn}{CP($N-1$)$\times$C}

\newcommand{\lar}{\lambda_R}
\newcommand{\lal}{\lambda_L}
\newcommand{\larl}{\lambda_{R,L}}
\newcommand{\lalr}{\lambda_{L,R}}
\newcommand{\blar}{\ov{\lambda}{}_R}
\newcommand{\blal}{\ov{\lambda}{}_L}
\newcommand{\blarl}{\ov{\lambda}{}_{R,L}}
\newcommand{\blalr}{\ov{\lambda}{}_{L,R}}

\newcommand{\bgamma}{\ov{\gamma}}
\newcommand{\bpsi}{\ov{\psi}{}}
\newcommand{\bphi}{\ov{\phi}{}}
\newcommand{\bxi}{\ov{\xi}{}}

\newcommand{\ff}{\mc{F}}
\newcommand{\bff}{\ov{\mc{F}}}

\newcommand{\eer}{\epsilon_R}
\newcommand{\eel}{\epsilon_L}
\newcommand{\eerl}{\epsilon_{R,L}}
\newcommand{\eelr}{\epsilon_{L,R}}
\newcommand{\beer}{\ov{\epsilon}{}_R}
\newcommand{\beel}{\ov{\epsilon}{}_L}
\newcommand{\beerl}{\ov{\epsilon}{}_{R,L}}
\newcommand{\beelr}{\ov{\epsilon}{}_{L,R}}

\newcommand{\bi}{{\bar \imath}}
\newcommand{\bj}{{\bar \jmath}}
\newcommand{\bk}{{\bar k}}
\newcommand{\bl}{{\bar l}}
\newcommand{\bm}{{\bar m}}

\newcommand{\nz}{{n^{(0)}}}
\newcommand{\no}{{n^{(1)}}}
\newcommand{\bnz}{{\ov{n}{}^{(0)}}}
\newcommand{\bno}{{\ov{n}{}^{(1)}}}
\newcommand{\Dz}{{D^{(0)}}}
\newcommand{\Do}{{D^{(1)}}}
\newcommand{\bDz}{{\ov{D}{}^{(0)}}}
\newcommand{\bDo}{{\ov{D}{}^{(1)}}}
\newcommand{\sigz}{{\sigma^{(0)}}}
\newcommand{\sigo}{{\sigma^{(1)}}}
\newcommand{\bsigz}{{\ov{\sigma}{}^{(0)}}}
\newcommand{\bsigo}{{\ov{\sigma}{}^{(1)}}}

\newcommand{\rrenz}{{r_\text{ren}^{(0)}}}

\begin{document}

%%%%%%%%%%%%%%%%%%%%%%%%%%%%%%%%%%%%%%%%%%%%%%%%%%%%%%%%%%%%%%%%%%%%%%%%%%%%%%%%%%
%
%	  		        S E C T I O N
%
%%%%%%%%%%%%%%%%%%%%%%%%%%%%%%%%%%%%%%%%%%%%%%%%%%%%%%%%%%%%%%%%%%%%%%%%%%%%%%%%%%
\section{1-loop Effective Action}

We take the Lagrangian of the heterotic \CPC sigma model with twisted masses as
\begin{align}
\label{sigma_full}
% 
\notag
 	\mc{L}_\text{1+1} & ~~=~~
	-\,\frac{1}{8e^2}\,F_{RL}^2  ~+~ \frac{1}{e^2} \left|\p_k \sigma\right|^2 
	~+~ \frac{1}{2e^2}\, D^2
	~+~ \frac{1}{e^2}\, \blar\, i\p_L\, \lar  ~+~  \frac{1}{e^2} \blal\, i\p_R\, \lal
	\\[2mm]
%
\notag
	&~
	~+~ 
	\left| \nabla n \right|^2  ~+~ 2 \Bigl| \sigma - \frac{m^l}{\sqrt 2} \Bigr|^2 \left| n^l \right|^2
	~+~ iD \left( \left|n^l \right|^2 - r_0 \right)
	\\
%
\notag
	&~
	~+~ \bxir\, i\nabla_L \xir  ~+~ \bxil\, i\nabla_R \xil ~+~
	i\sqrt{2}\, \Bigl( \sigma - \frac{m^l}{\sqrt{2}} \Bigr) \ov{\xi}{}_{Rl} \xi_L^l
	~+~ i\sqrt{2}\, \Bigl( \ov{\sigma} - \frac{\ov{m}{}^l}{\sqrt{2}} \Bigr) \ov{\xi}{}_{Ll} \xi_R^l
	\\[2mm]
%
	&
	~+~ i\sqrt{2}\, \ov{\xi_{[R}\, \lambda}{}_{L]}\, n
	~-~ i\sqrt{2}\, \nbar\,  \lambda_{[R}\, \xi_{L]}
	\\[4mm]
%
\notag
	&~
	~+~ \bzr\, i\p_L\, \zr   ~+~   \bff\, \ff
	\\[4mm]
%
\notag
	&~
	~-~ 2i\, \omega \cdot \blal\, \zr
	~-~ 2i\, \ov{\omega} \cdot \bzr\, \lal
	~+~ 2i\, \omega \cdot \ff\, \sigma
	~+~ 2i\, \ov{\omega} \cdot \bff\, \ov{\sigma}\,,
	\\[3mm]
%
\notag
	&~
	l~=~1,...\,N\,.
\end{align}
	We have defined here
\[
	\omega ~~=~~ \sqrt{r_0}\,\delta\,, \qquad\qquad\qquad r_0 ~~=~~ 2\beta\,.
\]

	To build the low-energy effective action, we integrate over all but one $ n^l $ and $ \xi^l $.
	We also ignore the gauge field $ A_\mu $.
	Variables $ n^i $ and $ \xi^i $, $ i = 1,...\, N-1 $, enter the Lagrangian quadratically,
\begin{align*}
%
	\mc{L} & ~~\supset~~ \nbar{}_i 
		   \lgr -\, \p_k^2 ~+~ 2\,\Bigl| \sigma - \frac{m^i}{\sqrt{2}} \Bigr|^2 ~+~  i\, D \rgr n^i
	\\
%
	&~ ~+~ \lgr \bxi_{Ri}~~~~~  \bxi_{Li} \rgr 
		\lgr \begin{matrix}
			i\,\p_L  &  
			i\,\sqrt{2}\, \Bigl(\sigma - \frac{\displaystyle m^i}{\displaystyle \sqrt{2}}\Bigr) \\
			i\,\sqrt{2}\,\Bigl(\ov{\sigma} - \frac{\displaystyle \ov{m}{}^i}{\displaystyle\sqrt{2}}\Bigr) &  
			i\,\p_R 
		     \end{matrix} \rgr
		\lgr \begin{matrix}
			\xi_R^i \\[5mm] \xi_L^i
		     \end{matrix} \rgr.
\end{align*}
	Integrating over these variables produces the determinant
\[
	\prod_{i=1}^{N-1}
	\frac{ \det \lgr -\,\p_k^2 ~+~ 
			2\,\Bigl| \sigma - \frac{\displaystyle m^i}{\displaystyle \sqrt{2}} \Bigr|^2 \rgr }
             { \det \lgr -\,\p_k^2 ~+~ 
			2\,\Bigl| \sigma - \frac{\displaystyle m^i}{\displaystyle \sqrt{2}} \Bigr|^2 ~+~ i\, D \rgr } \,.
\]
	Evaluating this determinant at one loop, and denoting $ n^N \equiv n $, and its mass
	$ m_0 \equiv m^N $, one arrives at the effective potential
\begin{align}
%
\notag
	V_\text{eff} & ~~=~~  \int d^2x 
		\Biggl\lgroup  \lgr iD ~+~ 2\,\Bigl|\sigma - \frac{m_0}{\sqrt{2}}\Bigr|^2 \rgr |n|^2 
	\\
%
\label{Veff}
	&\qquad\qquad\quad
	-~ 
	\frac{1}{4\pi}\, \sum_{i=1}^{N-1} \Bigl( iD ~+~ 2\,\Bigl|\sigma - \frac{m^i}{\sqrt{2}}\Bigr|^2 \Bigr)\,
		\ln\, \frac{ iD \,+\, 2 | \sigma - \frac{\displaystyle m^i}{\displaystyle \sqrt{2}} |^2} {\Lambda^2}
	\\
%
\notag
	&\qquad\qquad\quad
	+~
	\frac{1}{4\pi}\, \sum_{i=1}^{N-1} 2\,\Bigl|\sigma - \frac{m^i}{\sqrt{2}}\Bigr|^2\,
			\ln\, \frac{ 2\, | \sigma - \frac{\displaystyle m^i}{\displaystyle \sqrt{2}} |^2 } { \Lambda^2 }
	~+~ 
	\frac{1}{4\pi}\, iD\, (N-1) 
	\\[1mm]
%
\notag
	&\qquad\qquad\quad
	+~
	4\, |\omega|^2\,\bigl|\sigma\bigr|^2 \Biggr\rgroup.
\end{align}

	Minimizing the potential \eqref{Veff} with respect to $n$, $D$ and $ \sigma $, one arrives at
	the vacuum equations
\begin{align}
%
\label{eff1}
	&
	|n|^2  ~~-~~  \frac{1}{4\pi} \sum_i^{N-1}\, 
		{\rm log}\, 
		\frac{i D ~+~ 2\, \Bigl| \sigma -  \frac{\displaystyle m^i}{\displaystyle \sqrt{2}} \Bigr|^2}
							{\Lambda^2}   ~~=~~ 0 \\				
%
\label{eff2}
	&
	\lgr i D ~~+~~ 2\,\Bigl|\sigma - \frac{m_0}{\sqrt{2}}\Bigr|^2 \rgr n ~~=~~ 0 \\
%
\label{eff3}
&
	\Bigl(\sigma - \frac{m_0}{\sqrt{2}}\Bigr)|n|^2 
		~-~ 
	\frac{1}{4\pi} \sum_i^{N-1}\,
			\Bigl(\sigma - \frac{m^i}{\sqrt{2}}\Bigr)\,
		{\rm log}\, 
		\frac{i D ~+~ 2\, \Bigl| \sigma \,-\, \frac{\displaystyle m^i}{\displaystyle \sqrt{2}} \Bigr|^2}
		{2\, \Bigl| \sigma - \frac{\displaystyle m^i}{\displaystyle \sqrt{2}} \Bigr|^2}
		~+~ 2\, |\omega|^2\sigma ~~=~~ 0\,.
\end{align}

	Immediately we observe that Eq.~\eqref{eff2} implies two distinct cases,
\begin{align}
%
\label{higgsph}
& \bullet \qquad
	 iD ~+~ 2\,\Bigl|\sigma - \frac{\displaystyle m_0}{\displaystyle \sqrt{2}}\Bigr|^2 ~~=~~ 0  
	\qquad\qquad\qquad\qquad\qquad\qquad\qquad\qquad
\\[1mm]
%
\notag
& \qquad \text{and} \\[2mm]
%
\label{strongph}
& \bullet \qquad
	 n ~~=~~ 0 \,. 
\end{align}
	These cases correspond to the Higgs and the strong-coupling phases of the theory. 
	We solve these equations perturbatively, assuming $ |\omega|^2 $ a small parameter,
\begin{align*}
%
	n & ~~=~~ \nz  ~~+~~ |\omega|^2\,\no ~~+~~ \ldots\,, \\[2mm]
%
	iD & ~~=~~ i\Dz  ~~+~~ |\omega|^2\,i\Do ~~+~~ \ldots\,, \\[2.3mm]
%
	\sigma & ~~=~~ \sigz ~~+~~ |\omega|^2\,\sigo ~~+~~ \ldots\,.
\end{align*}
	Here $ \nz $, $ \Dz $ and $\sigz$ constitute the solution of the \ntwot CP($N-1$) sigma model,
	in particular $ \Dz = 0 $ in both phases.

	To obtain simple expressions for the solution we will assume the masses to be sitting
	on the circle,
\[
	m^k ~~=~~ m \cdot e^{i 2\pi k / N }\,, \qquad\qquad k ~=~ 0,...\, N-1\,.
\]

%%%%%%%%%%%%%%%%%%%%%%%%%%%%%%%%%%%%%%%%%%%%%%%%%%%%%%%%%%%%%%%%%%%%%%%%%%%%%%%%%%
%
%	  		        S E C T I O N
%
%%%%%%%%%%%%%%%%%%%%%%%%%%%%%%%%%%%%%%%%%%%%%%%%%%%%%%%%%%%%%%%%%%%%%%%%%%%%%%%%%%
\section{Higgs Phase}

	The large-$N$ supersymmetric solution of the \ntwot CP($N-1$) sigma model
	in the Higgs phase is
\begin{align*}
%
	\nz & ~~=~~ \sqrt{\rrenz}\,,   \qquad\qquad\qquad\text{\it the phase of $\nz$ is not determined}\\[3mm]
%
	i\Dz & ~~=~~ ~~~\,0\,, \\[2mm]
%
	\sigz & ~~=~~ ~~\frac{m_0}{\sqrt{2}}\,,
\end{align*}
	where $ \rrenz $ is the renormalized coupling of the unperturbed theory,
\[
	\rrenz ~~=~~ \frac{N}{2\pi}\,{\rm log}\,m/\Lambda\,.
\]

	Expanding equations \eqref{eff1}-\eqref{eff3} to the first order in $|\omega|^2$, we obtain
\begin{align}
%
\notag
	&
	\bnz\,\no ~+~ \text{h.c.} ~~=~~
	\frac{1}{4\pi}\, \sum_{i}^{N-1}\, 
		\frac{ i\Do ~+~ 2\lgr \bsigz\, \sigo ~-~ \bsigo\,\frac{\displaystyle m^i}{\displaystyle \sqrt{2}}
				~+~ \text{h.c.} \rgr }
			{ 2 \Bigl| \sigz - \frac{\displaystyle m^i}{\sqrt{\displaystyle 2}} \Bigr|^2 }\,, \\
%
\label{higgseq}
	&
	i\Do ~~=~~ 0\,, \\[2mm]
%
\notag
	&
	\sigo\,|\nz|^2  ~-~ 
	\frac{1}{4\pi}\,i\Do\, \sum_{i}^{N-1}\, \frac{1}
				{2\Bigl( \bsigz - \frac{\displaystyle \ov{m}{}^i}{\displaystyle \sqrt{2}} \Bigr) }
	~+~ 2\,\sigz ~~=~~ 0\,.
\end{align}
	We can see that $ iD $ vanishes to the first order in $ |\omega|^2 $.
	Thus, for $ iD $ the first order of expansion is not sufficient, since we know that supersymmetry
	is broken and hence $ iD ~\neq~ 0 $.
	This variable, however, is easy to recover in the Higgs phase given the corresponding expansion of $ \sigma $
	via Eq.~\eqref{higgsph}.

	The solution to equations \eqref{higgseq} can be written as
\begin{align*}
%
	i \Dz & ~~=~~ 0\,,                     &       i\Do & ~~=~~ 0\,,  
			\qquad\qquad iD^{(2)} ~~=~~ -\,2\,|\sigo|^2\,, \\[3mm]
%
	\sigz & ~~=~~ \frac{m_0}{\sqrt{2}}\,,  &       \sigo & ~~=~~ -\,\frac{2\sigz}{|\nz|^2}\,, \\[3mm]
%
	|\nz|^2 & ~~=~~ \rrenz\,,              &    
			\no & ~~=~~ -\,\frac{2\,m^0}{\bnz\,|\nz|^2}\,\frac{1}{4\pi}\,
					\sum_{i}^{N-1} \frac{1}{m^0 - m^i}\,.
\end{align*}	

	To simplify these expressions, as noted before, we put the masses on the circle, which gives then
\[
	\sum_{i=1}^{N-1} \frac{1}{m^0 - m^i} ~~=~~ \frac{N-1}{2m} ~~=~~ \frac{N}{2m} ~~+~~ O(1)\,.
\]
	Finally, we obtain 
\begin{align*}
%
	\sigma & ~~=~~ \frac{m^0}{\sqrt{2}}\, \lgr 1 ~~-~~ \frac{2\,|\omega|^2}{|\nz|^2} \rgr ~~+~~ \ldots \,,
	\\[2mm]
%
	i D & ~~=~~ -\, 4\, \frac{m_0^2}{(\rrenz)^2}\, |\omega|^4 ~~+~~ \ldots\,,
	\\[2mm]
%
	n & ~~=~~ \sqrt{\rrenz} ~~-~~ \frac{N}{4\pi}\,\frac{1}{\rrenz\,\bnz}\,|\omega|^2 ~~+~~ \ldots
\end{align*}
	for the Higgs phase, where
\[
	\rrenz ~~=~~ \frac{N}{2\pi}\,{\rm log}\,m/\Lambda\,.
\]

%%%%%%%%%%%%%%%%%%%%%%%%%%%%%%%%%%%%%%%%%%%%%%%%%%%%%%%%%%%%%%%%%%%%%%%%%%%%%%%%%%
%
%	  		        S E C T I O N
%
%%%%%%%%%%%%%%%%%%%%%%%%%%%%%%%%%%%%%%%%%%%%%%%%%%%%%%%%%%%%%%%%%%%%%%%%%%%%%%%%%%
\section{Strong Coupled Phase}

	The zeroth order in $ |\omega|^2 $ solution is 
\begin{align*}
%
	\nz & ~~=~~ 0\,, \\[2mm]
%
	i \Dz & ~~=~~ 0\,, \\[1mm]
%
	\sigz & ~~=~~ \frac{\wt{\Lambda}}{\sqrt 2}\cdot e^{i\frac{2\pi l}{N}}\,,  
	\qquad\qquad\qquad
	\wt{\Lambda} ~~=~~ \sqrt[N]{ \Lambda^N ~+~ m^N }\,,
\end{align*}
	for some fixed $ l = 0,...\,N-1 $.
	Furthermore, in this phase  $ n $ is known exactly,
\[
	n ~~=~~ 0\,.
\]
	The rest two equations in \eqref{eff1}-\eqref{eff3} give at the first order in $ |\omega|^2 $
\begin{align*}
%
	&
	\sum_{i}^{N-1}\, \frac{ i \Do ~+~ 2 \lgr \bsigz\,\sigo ~-~ 
					\sigo\, \frac{\ov{\displaystyle m}{}^i}{\sqrt{\displaystyle 2}}
				~+~ \text{h.c.} \rgr }
			    { 2\, \Bigl| \sigz - \frac{\displaystyle m^i}{\sqrt{\displaystyle 2}} \Bigr|^2 }
			~~=~~ 0\,,
	\\[2mm]
%
	&
	\frac{1}{4\pi}\, \sum_i^{N-1}\, \frac{i \Do}
				    {2\,\Bigl( \bsigz - \frac{\displaystyle\ov{m}{}^i}{\displaystyle \sqrt{2}} \Bigr) }
			~~=~~ 2\, \sigz\,.
\end{align*}

	The solution to these equations are given by
\begin{align}
%
\notag
	n & ~~=~~ 0 \,,  \qquad\qquad\qquad\qquad i \Dz ~~=~~ 0\,,
	\\[3mm]
%
\label{strongeq}
	i \Do & ~~=~~ 8\pi \cdot \frac{ 2\sigz } 
			{\displaystyle \sum_{i}^{N-1} 
				\frac{1}{ \bsigz - \ov{m}{}^i/\sqrt{2}} } \,,
	\qquad\qquad\qquad i ~=~ 1,...\, N-1\,,
	\\[-8mm]
%
\notag
	\sigo & \cdot \sum_{i}^{N-1} \frac{1}
			{ \sigz - \frac{\displaystyle m^i}{\displaystyle\sqrt{2}} }
		~+~ \text{h.c.} ~~=~~
	-\, 8\pi \cdot \sigz \cdot 
	\frac{ \displaystyle \sum_{i}^{N-1} 
			\frac{1}{ \Bigl| \sigz - \frac{\displaystyle m^i}{\displaystyle\sqrt{2}} \Bigr|^2 } }
	  { \displaystyle \sum_{i}^{N-1} 
				\frac{1}{ \bsigz - \frac{\displaystyle \ov{m}{}^i}
							{\displaystyle \sqrt{2}} } }\,.
\end{align}

	We use the following relations to simplify the above solution in the case when the masses 
	are distributed on a circle,
\begin{align*}
%
	&
	\sum_{k=0}^{N-1} \frac{1} { 1 \,-\, \alpha\,e^{\frac{2\pi i k}{N}} } ~~=~~
		\frac{N}{1 ~-~ \alpha^N}\,,
	\\
%
	&
	\sum_{k=0}^{N-1} \frac{1} { (1 + \alpha^2) ~-~ 2\,\alpha \cos \frac{2 \pi k}{N} }
		~~=~~
	\frac{1} {1 \,-\, \alpha^2} 
	\lgr \frac{2\,N}{ 1 \,-\, \alpha^N } ~-~ N \rgr .
\end{align*}

	This gives,
\begin{align*}
%
	& 
	\sum_{k=1}^{N-1} \frac{ 1 }
		{ \sigz - \frac{\displaystyle m^k}{\displaystyle \sqrt{2}} } ~~=~~ 
		 - \, \frac{ 1 }
		{ \sigz - \frac{\displaystyle m}{\displaystyle \sqrt{2}} } ~+~
		N\, \frac{\wt{\Lambda}{}^N}{\Lambda^N}\, \frac{1}{\sigz} \,,
	\\
%
	&
	\sum_{k=1}^{N-1} \frac{ 1 }
                { \Bigl| \sigz - \frac{\displaystyle m^k}{\displaystyle \sqrt{2}} \Bigr|^2 } ~~=~~
	-\, \frac{1}
	{ \Bigl| \sigz - \frac{\displaystyle m}{\displaystyle \sqrt{2}} \Bigr|^2 } 
	~+~
	\frac{2\,N}{ \wt{\Lambda}{}^2 - m^2 } \cdot
	\frac{ \wt{\Lambda}{}^N + m^N } { \Lambda^N }\,.
\end{align*}

	In fact we will only need the leading-N contribution from them above relations.
	In particular, if we substitute these relations into Eq.~\eqref{strongeq} directly,
	we will not have $ iD $ real.
	The reason {\it perhaps} is that the $ \sigz $ solution is only valid up to $ O(1/N) $
	contributions.
	Therefore, we ignore the O(1) contributions versus O(N).
\begin{align*}
%
	i \Do & ~~=~~ \frac{16\pi}{N}\, \frac{\Lambda^N}{\wt{\Lambda}{}^N}\, |\sigz|^2\,,
	\\[3mm]
%
	| \sigo | & ~~=~~
	-\, \sqrt{2}\,\frac{2\pi}{N}\, \frac{\Lambda^N}{\wt{\Lambda}{}^N}\,
		\frac{ \wt{\Lambda}{}^3 } {\wt{\Lambda}{}^2 - m^2 }\,
		\Bigl[ 1 ~+~ \frac{m^N}{\wt{\Lambda}{}^N} \Bigr] \cdot
		\Bigl( \cos \bigl( \frac{2\pi l }{N} - \varphi \bigr) \Bigr)^{-1}\,,
\end{align*}
	where $ \varphi $ is the arbitrary phase of $ \sigo $.
	We can choose this phase to be the same as that of $ \sigz $, 
\[
	\varphi ~~\equiv~~ \frac{2\pi l}{N}
\]
	for the cosine to disappear.
	If we take the limit of small masses
\[
	\frac{m^N}{\Lambda^N} ~~\ll~~ 1\,,
\]
	and also note that
\[
	\wt{\Lambda} ~~=~~ \Lambda 
\]
	with exponential accuracy in $ 1/N $,
	we can further simplify the result, and arrive at
\begin{align*}
%
	n & ~~=~~ 0 
	\\[3mm]
%
	i D &  ~~=~~ |\omega|^2 \cdot \frac{8\pi}{N}\cdot \Lambda^2 ~+~ \ldots \,, 
	\qquad\qquad\qquad \text{for }\frac{m^N}{\Lambda^N} ~\ll~ 1\,,
	\\[3mm]
%
	\sigma & ~~=~~ \frac{\Lambda}{\sqrt 2} \cdot e^{\frac{2\pi i l}{N}} 
			~-~ |\omega|^2\, \sqrt{2}\, \frac{2\pi}{N}\, 
				\frac{\Lambda^3}{\Lambda^2 - m^2}\, e^{\frac{2\pi i l}{N}}
	\,.
\end{align*}

	Note that although $ |\omega|^2 $ grows as $ O(N) $ for large $N$, the coefficients of 
	$|\omega|^2$-corrections are suppressed by the corresponding power of $ 1/N $ so that 
	the corrections are neutral in $ N $.

%%%%%%%%%%%%%%%%%%%%%%%%%%%%%%%%%%%%%%%%%%%%%%%%%%%%%%%%%%%%%%%%%%%%%%%%%%%%%%%%%%
%%%%%%%%%%%%%%%%%%%%%%%%%%%%%%%%%%%%%%%%%%%%%%%%%%%%%%%%%%%%%%%%%%%%%%%%%%%%%%%%%%
%
%                            B I B L I O G R A P H Y
%
%%%%%%%%%%%%%%%%%%%%%%%%%%%%%%%%%%%%%%%%%%%%%%%%%%%%%%%%%%%%%%%%%%%%%%%%%%%%%%%%%%
%%%%%%%%%%%%%%%%%%%%%%%%%%%%%%%%%%%%%%%%%%%%%%%%%%%%%%%%%%%%%%%%%%%%%%%%%%%%%%%%%%
\small
\begin{thebibliography}{99}
\itemsep -2pt



\bibitem{Edalati}
  M.~Edalati and D.~Tong,
  %``Heterotic vortex strings,''
  JHEP {\bf 0705}, 005 (2007)
  [arXiv:hep-th/0703045].
  %%CITATION = JHEPA,0705,005;%%

\bibitem{SYhet}
  M.~Shifman and A.~Yung,
  %``Heterotic Flux Tubes in N=2 SQCD with N=1 Preserving Deformations,''
  Phys.\ Rev.\  D {\bf 77}, 125016 (2008)
  [arXiv:0803.0158 [hep-th]].
  %%CITATION = PHRVA,D77,125016;%%

\bibitem{BSYhet}
  P.~A.~Bolokhov, M.~Shifman and A.~Yung,
  %``Description of the Heterotic String Solutions in U(N) SQCD,''
  [arXiv:0901.4603 [hep-th]].
  %%CITATION = ARXIV:0901.4603;%%
  
\bibitem{SVZw}
  M.~Shifman, A.~Vainshtein and R.~Zwicky,
  %``Central charge anomalies in 2D sigma models with twisted mass,''
  J.\ Phys.\ A  {\bf 39}, 13005 (2006)
  [arXiv:hep-th/0602004].
  %%CITATION = JPAGB,A39,13005;%%


\end{thebibliography}


\end{document}
